%%%%%%%%%%%%%%%%%%%%%%%%%%%%%%%%%%%%%%%%%%%%%%%
%%%%%%%%%%%%%%%%%%%%%%%%%%%%%%%%%%%%%%%%%%%%%%%
%%%%%%%%%%%%%%%%%%%%%%%%%%%%%%%%%%%%%%%%%%%%%%%
\setcounter{section}{1}\section{課題のまとめ}
\label{c10.s2}
\paragraph{はじめに}

以上の日本のサイエンス達成のために必要な要求を実現していくために、以下の様な科学検討課題ならびにこれからでも参入可能な技術開発課題などがある。これらを今後解決していくことが望まれる。

\paragraph{科学検討課題}

各科学領域では天文学的・宇宙物理学的な課題がまとめられたが、その解決の達成がもっとも期待される観測の具体的な内容の検討はまだ十分とは言えない。例えばどの視野や天体を観測するかのサーベイ戦略を具体化していく検討が必要だろう。また限られた資源の中で、サーベイの効率化を領域をまたがって検討するべきである。一例を挙げるならば、ContinuumとPolarimetryは親和性が高く、SKA試験機によるサーベイでも相乗り観測についての調整が長く議論されてきている。さらには、観測において単一のゴールではなく複数のゴールを見定める検討が必要である。それは観測のミニマムサクセス・ベストサクセス達成の順位付けをすることにも繋がる。領域をまたがった日本の優先すべきキーサイエンスの検討も必要かもしれない。加えて、SKAが動き出した時に、すぐに観測を実施しデータ解析を始められるような体制づくりも重要である。例えば既存の観測装置によりパイロット観測を行ったり、アーカイブデータの解析を行うなどの取り組みも必要だろう。最後に、一般的な科学計画と同じく、解説マニュアルの作成などの各種の整備も重要だろう。これにより分野外の研究者の参画をしやすくし、また次世代を担う若手の育成に役立てる。

\paragraph{技術開発課題}

技術開発課題に関しては、SKAOとの情報共有が始まっている。それにより分かっている共通の課題は、あらゆる装置の高信頼性・低電力・低コストである。具体的な要素としては次のようなものがある。
\begin{itemize}
\item {\bf 冷却システム}\\
冷却システムは運用コストを左右するので、低電力・低コストのソリューションが必要である。かつ、南アフリカやオーストラリアの荒野でも使える高信頼性が求められる。とりわけPAFはその大きさが例えばBand2で1.8m直径もある。Band1はこれよりさらに大きい。これを均一に定常的に、一区切り5年間という期間、ずっと冷却し続けられるだけのものがなければならない。
\item {\bf ビーム形成器}\\
特に開口アンテナビーム形成は開発のまっただ中にある。LOFARやMWAなどではビームの安定性について数々の課題が報告されている。またビーム形成器は比較的電力消費の高い要素である。SKA1-LOWだけでも3メガワット近くを消費する。
\item {\bf システムRFIの軽減}\\
特に長波長側ではシステムRFIをできるだけ下げることが要求感度達成に求められる。
\item {\bf PAF}\\
PAFはASKAPにて実証が進んでいる段階であるが、さらに経験を積む必要がある。小型フィードの集積であるPAFシステムの安定性や有効視野の向上、マルチビーム形成の最適化が待たれる。PAFはサーベイ速度を劇的に増加することのできる魅力的な潜在能力を秘めているため、実現が大変望まれる。
\item {\bf WBSPF}\\
AIPであるWBSPFは開発そのものが課題である。これが実現すれば、複数帯域の受信のために複数設置しなければならなかった冷却システムや変換器といった機器を減らすことができ、高信頼・低電力・低コストに確実に寄与する。そして同じ視野を異なる周波数帯域のために何度も走査する必要がなくなる。サーベイ速度を劇的に増加することのできる魅力的な潜在能力を秘めているため、実現が大変望まれる。
\item {\bf MFAA}\\
AIPであるMFAAも開発の全てが課題といっても過言ではない。SKAでの実用に直接参考になるような実証がまだないため、開発が急務である。実現されれば200平方度という驚異的な視野を得られるとも言われているため、実現が大変望まれる。
\item {\bf CSP}\\
CSPでは高速なGPGPUやFPGAの相関器の開発はもちろんのこと、高信頼・低電力・低コストのソリューションも望まれる。ソフト面でもアルゴリズムのさらなる改善ができれば、コスト削減に確実につながるだろう。
\item {\bf ビッグデータ}\\
大量のデータが得られるSKAはまさにビッグデータの課題に直面している。データを収集、転送、保存、処理、管理するソリューションの開発が必要である。とりわけ100万あるLFAAのデータ転送の実現にはさまざまな可能性があるだろう。RFファイバーリンクやデジタル信号処理の分野で、高信頼性・低電力・低コストのソリューションが必要である。また人の手で処理しきれないビッグデータの活用という点も、次のSDPとも関係して重要である。
\item {\bf SDP}\\
膨大な数のデータはもはや人の手で一つ一つ解析することができない。ゆえにサイエンスの達成に必要な科学データ処理を自動的に処理するパイプラインを構築しなければならない。LFAAでは較正作業において課題が報告されている。視野が広いため大気補正が難しく、背景や前景の取り除きの高精度化が待たれている。
\end{itemize}

\paragraph{結語}

以上のことから、今後も様々な検討と開発を組織的・戦略的に進めていかなければならないことは明らかである。これらの準備活動自体が、国内の科学技術水準を高く維持するのに寄与するだろう。
