%%%%%%%%%%%%%%%%%%%%%%%%%%%%%%%%%%%%%%%%%%%%%%%
%%%%%%%%%%%%%%%%%%%%%%%%%%%%%%%%%%%%%%%%%%%%%%%
%%%%%%%%%%%%%%%%%%%%%%%%%%%%%%%%%%%%%%%%%%%%%%%
\setcounter{section}{0}\section{要求のまとめ}
\label{c10.s1}

\paragraph{はじめに}

本書ではSKA計画で実施可能な宇宙論、宇宙再電離、銀河進化、パルサー、宇宙磁場、近傍宇宙時空計測、星間物質、そして突発天体の科学的課題について議論した。各領域では、まず科学課題を理解するのに必要な基礎的な背景知識を紹介した。専門家だけでなく非専門家でも、それぞれの領域のこれまでの知見と未解決の課題について把握できるよう配慮した。次に、国際SKAサイエンスブック2015についてレビューを行った。世界の最先端が何処にあるのか、世界がSKA計画のどこに注目しているかについて解説した。そして最後に、日本がSKAを用いて果たせるであろう研究の独自性を議論した。これにより日本がリードできそうな研究課題を明確にし、日本がSKA計画に参加する科学的動機がどの程度あるのかを明らかにした。その結果、日本には幅広い分野のセンチ波・メートル波への期待があることは明らかである。そして真に革新的なサイエンスを次の10年や20年の間に達成するためには、以下の様な科学的要求と大枠としての解決手段があることがわかった。

\paragraph{高感度の要求と解決手段}

高い感度は、同じ天体を観測する時間を短縮し、まだ見えない暗い天体までを見つけ出すために必要な本質的性能である。それは日本で検討されているサイエンスの全てにおいて望まれている性能であった。特に宇宙再電離の21cm線を見つけ出すこと、特別な条件のパルサーを見つけ出すこと、膨大な数の系外天体を見つけ出すこと、そして未知の天文現象を発見することなどである。高感度を実現するために感度指標$A_{\rm eff}/T_{\rm sys}$をいかに大きくするかが課題である。システム雑音温度$T_{\rm sys}$を桁で下げることには限界があるため、有効開口面積$A_{\rm eff}$を桁で上げるための開口アンテナ100万台程度、パラボナアンテナ2000台程度が要求される。

\paragraph{高角度分解の要求と解決手段}

より高い空間分解能力も日本で検討されているサイエンスの全てにおいて望まれている性能であった。センチ波・メートル波においては、1 GHzで$1''$を大幅に切るレベルが、HIのマッピングや宇宙磁場の研究においてマイルストーンである。その結果、SKAの仕様として最大基線長$B_{\rm max}$は3000 km程度が要求される。長い基線長はステーションを分けたVLBIの可能性ももたらし、位置天文や銀河・AGNジェット研究などからの期待も大きい。

\paragraph{広視野の要求と解決手段}

突発現象においては、感度と視野は代替し合わない。より多くの突発現象を捉えるには、究極的には全天モニターが望ましい。一般の観測においても視野が広いほうがオーバーヘッドは少ないため、観測の効率も良いと言える。センチ波・メートル波では波長が長いため、ミリ波・サブミリ波に比べ視野が広いという特徴がある。特にパラボラアンテナについては比較的小口径のアンテナとすることによって、視野を広くとることを可能とする。さらなる視野拡大のためAdvanced Instrumental Program (AIP)としてパラボラアンテナ用のPhased Array Feed (PAF)や開口アンテナを用いたBeam Formingなどの開発が要求される。

\paragraph{広帯域の要求と解決手段}

日本では宇宙再電離研究の動機が強く、ゆえに100 MHz以下にまでシームレスに網羅することが強く望まれる。また銀河進化、星間物質や宇宙生命などでは多くの輝線から物質量や化学進化をさぐるために、10 GHz以上までの高周波帯域のデータが望まれる。宇宙磁場の研究においては、広帯域な偏波データを取得することで、全く新しい解析の方法論を実践できるようになると指摘された。その方法論を我が国の独自性と見据えていることからも、広帯域なデータを取得することが望まれる。広帯域を実現するためには、フィードおよび低雑音アンプの広帯域化、サンプラーの高速化、相関器機能の革命的な拡充が要求される。

\paragraph{日本の活動}

特に2010年に開催したSKA-JP WS2010では「広帯域」を日本のキーワードとして掲げた。特に開発項目の一つであるデジタル技術などは波長を問わず活かせる項目であり、ミリ波・サブミリ波望遠鏡やVLBIなど、これまで我が国でも開発が進められてきた技術を活かすことが可能である。実際これまでTapered Slot Antennaや多重構造ホーンアンテナなどのフィード開発、超高速アナログ-デジタル変換器、ROACHボードによる広帯域デジタル分光計の開発などが進められており、広帯域化は我が国が貢献可能な開発要素の一つとなりうる。

\paragraph{結語}

以上のことから、センチ波・メートル波の大望遠鏡は我が国の科学コミュニティーが必要としている観測装置であり、技術的にも最先端の開発項目が多々あり、国内の科学技術水準を高く維持する上でも重要である。

