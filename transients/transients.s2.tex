%%
%% Section 2
%%
\section{国際SKAのサイエンス}\label{transients.s2}
国際SKAサイエンスブックでは「The Transient Universe」という領域区分があり、突発天体に関する論文はその領域に集録されている。また周辺領域の研究としてパルサーに関する論文もあり、それらは「Fundamental Physics with Pulsars」という領域に集録されている。本節では「The Transient Universe」に集録されている12の論文について紹介していく。表\ref{c09.s2.t1}には12の該当論文をリストした。次小節からはそれぞれの概要を紹介していく。他の国際科学検討班のポリシーとは異なり、これらの論文は草稿段階から2015年1月まで非公開であった。そのため準備期間が無く、詳細な解説はできなかった。突発天体科学研究班では、今後これらの論文について、時間をかけてより深く検討を進める予定である。

\begin{table}[htbp]
\begin{center}
\caption{国際SKAサイエンスブックの突発天体領域論文一覧$^*$}
\small
\begin{tabular}{llp{28zw}}
\hline\hline
\noalign{\smallskip}
ID$^\dag$ & PI & Title\\
\noalign{\smallskip}
\hline
\noalign{\smallskip}
The Transient Universe &&\\
051 & R. Fender & The Transient Universe with the Square Kilometre Array \\
052 & D. Burlon & The SKA View of Gamma-Ray Bursts \\
053 & S. Corbel & Incoherent transient radio emission from stellar-mass compact objects \\
054 & I. Donnarumma & SKA as a powerful hunter of jetted Tidal Disruption Events \\
055 & J.P. Macquart & Fast Transients at Cosmological Distances with the SKA \\
056 & L. Amati & The SKA contribution to GRB cosmology \\
058 & H.E. Bignall & Time domain studies of Active Galactic Nuclei with the Square Kilometre Array \\
060 & M. Perez-Torres & Core-collapse and Type Ia supernovae with the SKA \\
062 & T. O'Brien & Thermal radio emission from novae \& symbiotics with the Square Kilometre Array \\
064 & L. Wang & Investigations of supernovae and supernova remnants in the era of SKA \\
065 & P. Wilkinson & The Unknown Unknowns \\
066 & W. Yu & Early Phase Detection and Coverage of Extragalactic and Galactic Black Hole X-ray Transients with SKA \\
\noalign{\smallskip}
\hline
\noalign{\smallskip}
\multicolumn{2}{l}{\small $^{*}$ArXiv未投稿含む} & \\
\multicolumn{2}{l}{\small $^\dag$ PoS (AASKA14) ID} & \\
\end{tabular}\label{c09.s2.t1}
\end{center}
\end{table}

\subsection{SKAによる突発天体探査}\label{transients.s2.fender}
第~\ref{transients.s1}節および次節以降で述べるように、宇宙における突発現象は極めて高エネルギーな現象が元になっていると考えられ、その観測的研究によって新たな物理が開拓される可能性を秘めている。
ただし、突発現象は宇宙のいつどこで起こるかということが予測することができないため、探査のみをやみくもに続けることはリスクが高い。
しかしこのことは裏を返せば、突発天体の探査は他の目的の観測と並行して実施できるということである。
また突発現象は、初期の増光段階と時間経過後の減光段階で、電波放射の物理過程が異なることもあり、現象が発生した直後から継続して追観測することが重要である。
そのような特徴や課題をもつ突発現象の観測的研究では、以下の二つの機能をSKAに実装することが重要な鍵となる:
\begin{itemize}
	\item[(1)] 他の目的による観測と共存・並行して運用できる、突発天体の探査観測システム、
	\item[(2)] 突発天体が発見された際に、それを即時的・自動的に追観測するシステム。
\end{itemize}
本節ではこの二つの機能について紹介する。

%%
\subsubsection{SKAへの要求 (1): 他の観測と共存する突発天体探査システム}
%%
探査によって突発現象を発見できるかどうかは確率によって評価できるが、多かれ少なかれ運に左右される。
しかし一方で、探査は他の目的の観測と並行して実施することができ、多大な観測時間を費やすことができるという大きなメリットがある。
観測時間が多ければ多いほど突発現象を発見できる確率は上がるため、その探査システムを他の観測と共存できるように構築することで、効率的に探査ができるようになり大きな科学成果を期待できる。

そのようなシステムとして、例えば図~\ref{fig:transients.fender.commensal}に示すようなシステムをアフリカのMeerKATに実装することが提案されている (Armstrong et al., in prep)。
この共存システムは、観測データの処理経路を分岐させ、一方を本来の観測のために使用し、もう一方を突発天体のリアルタイム探査に使用するというものである。
図~\ref{fig:transients.fender.commensal}は観測データの流れと処理過程を示しており、図左上の相関器 (correlator) から処理が始まる。
通常の観測では、相関器から出力される電波干渉計のデータに対して図右方向に向かってさまざまな解析処理を施し、最終的に輝度分布画像を得る。
その観測には通常数時間以上を費やし、その長時間の観測データを後日取りまとめる場合が多い。
一方で突発天体は、その変動のタイムスケールが観測時間よりも短い可能性があり、またその追観測には即時性を要する。
そのため突発天体探査は通常観測だけでは不十分であり、リアルタイムにデータ処理を行う必要がある。
そこで通常観測と共存する形で処理経路を分岐させ、即座に簡易的な画像データを得て天体の光度変動を検出する。
それを行うのが図~\ref{fig:transients.fender.commensal}の左半分に示されるシステムである。
もし突発天体が検出されれば、自動的にその情報を VOEventNet と呼ばれる突発天体の発見速報ネットワークに通報する。
その通報によって、世界中の他の観測局がその突発天体を追観測することができれば、その起源や放射機構について詳しい情報が得られることだろう。
\begin{figure}
	\centering
	\includegraphics[width=12cm]{transients/transients.s2.fender.commensal.eps}
	\caption{他の観測と共存した突発天体探査システム。}
	\label{fig:transients.fender.commensal}
\end{figure}%

%%
\subsubsection{SKAへの要求 (2): 突発天体発見時の自動的な追観測システム}
%%
突発天体の観測的研究に重要なもう一方のシステムは、他の観測局によって発見された突発天体、あるいはSKA自身によって発見した突発天体を、自動的に追観測するというものである。
このようなシステムは既にイギリスのArcminute Microkelvin Imager Large Array (AMI LA) に実装されており、ガンマ線観測衛星Swiftで発見されたガンマ線バーストを約8時間後に追観測することに成功している \citep{2013MNRAS.428.3114S,2014MNRAS.440.2059A}。
この結果得られた電波帯域での光度曲線から、ガンマ線バーストのリバースショック (星間物質から噴出物の方向に伝わる衝撃波) による電波放射が、世界で初めて確認された。
同様にSwiftの速報によって、地球近傍の連星系 DG~CVn からのガンマ線フレアを6分以内に追観測した結果、電波帯域でも大きいフレアが確認され、広い周波数帯域でのコインシデンスが得られている。
しかし数日後には元の明るさに戻ってしまい、そのようなタイムスケールの短い突発現象を詳細に観測するためには、世界規模で即時的な追観測を行うことが重要である \citep{2015MNRAS.446L..66F}。
SKAにおいても同様のシステムを実装すれば、突発天体の研究にブレイクスルーを起こすことができるだろう。

%%
\subsubsection{まとめ}
%%
突発天体の観測的研究には、SKAに前述の二つのシステムを実装することが重要である。
SKAは広い視野と高い感度を兼ね備えており、さらにそのシステムが実装されれば、電波帯域での変動天体や突発天体を数多く発見することができるだろう。
その発見は、究極の宇宙物理を議論できる場所に、人類を導いてくれるはずである。
突発現象というのは宇宙の、とりわけ深宇宙の灯台であり、その観測によって新しい物理を開拓できるに違いない。


\input{transients/transients.s2.akahori.tex}
\subsection{未知の未知} \label{transients.s2.wilkinson}
アメリカ合衆国国防長官だったラムズフェルド氏の言葉 \citep{Rumsfeld} を借りると、物事には三種類あり、それらは
\begin{description}
	\setlength{\leftskip}{1cm}
	\item[Known knowns:] 我々が既に知っている事実、
	\item[Known unknowns:] 我々が知らないことを自覚している未知、
	\item[Unknown unknowns:] 我々が知らないことを自覚すらしていない未知、
\end{description}
というものである。
このことは、宇宙科学の分野においてもまったく同様である。
そこで本節では、unknown unknowns つまり未知の未知に相当する、予想すらできていない未知の天体現象の探査 (exploration of unknowns; EoU) について述べる。
とりわけ未知を発見するために持っておくべき哲学と、その哲学に基づいたSKAへの要求仕様を紹介する。

%%
\subsubsection{未知追究の哲学}
%%
宇宙科学における known knowns は既に研究されてきていることであるから、もしそれについて知りたければ文献を読めばよい。
一方 known unknowns について知りたければ、何を知るべきか把握していることなので、観測提案書を書いて観測を実施すればよい。
その観測が成功すれば、known unknowns は known knowns に変わり、宇宙科学は一歩前進することになるだろう。
しかしながら unknown unknowns については、我々はその存在にすら気付いておらず「知らないということすら知らない」。
そのため、どんな研究をしてどんな観測をすれば良いのか、検討するどころか見当すらつかない。
しかし過去の偉大な科学的発見が、そのような unknown unknowns を偶然発見したことに端を発しているのも事実である。

ゆえにSKAが長期に渡って科学成果を出し続けるためには、そのような unknown unknowns を発見できる能力を備えておかなくてはならない。
SKAは、前章までに述べてきたような「現在」未解決とされている問題に対して注力し、その解決を目標とする。
%それらの問題の解決は非常に重要であり、その実現によって宇宙科学は大きく前進することだろう。
しかしSKAが完成し隆盛を誇っているだろう2025年以降には、現在未解決のそれらの問題は既に解決されているに違いない。
その時SKAによる興奮はどこにあるかと言えば、それは解決済みの古い問題にはなく、新しい観測方式によって浮上するだろう新しい疑問、つまり unknown unknowns にある。
%人間はこの宇宙や構成物を第一原理から創り出せるほど、創造力も想像力もない。
それゆえ天文学では、我々が想像すらしていない「未知の何か」を見つけるための「備え」が重要であり、その備えに裏打ちされた「発見」が重要となる。

%%
\subsubsection{未知の探査 (EoU) のための要求}
%%
電波天文学における unknown unknowns がどのような天体現象なのかはもちろん予測できないが、「起源のわからない未知の突発現象」が既に発見されてきている。
%前述のとおり電波帯域における突発現象の中には、それを放射した実体がどのような天体なのか、あるいはどのような放射機構によって突発的な電波放射がなされたのか、わかっていないものが少なくない。
%第~\ref{transients.s1}節で述べたとおり、電波帯域における突発現象の中にはそれを放射した実体や放射機構がわかっていないものが多い。
%そのような未知の突発現象は、MHz から GHz という低周波帯域で、集光面積の大きな望遠鏡によって発見されていることが多い。
しかしそれらを発見した望遠鏡は、そのような未知の現象を捉えようとして設計されたわけではなかったし、何らかの理論をもとにして突発天体を探査したわけでもなかった。
ところが蓋を開けてみれば、未知の突発現象が次々と発見され、新たな科学が生まれている。
そしてこれからも、まだ予想もされていないような変動現象や突発現象が次々に見つかるに違いない。
それを効率的に発見できるのが、SKAである。

SKAは高い感度と広い視野を持っており、未知の現象を数多く発見できる潜在能力がある。
ただし潜在能力があったとしても、それを有効に活用できなければ未知の探査は進まないだろう。
SKAがその潜在能力を解放し未知の探査を効率的に行うためには、第~\ref{transients.s1}節で述べたような (1) 共存システムや (2) 自動追観測システムを活用できるような「柔軟性」が必要である。
%SKAがその潜在能力を解放し未知の探査を効率的に行うためには、次のようなシステムの柔軟性と拡張性が必要となる:
%\begin{itemize}
%	\item 小規模な試験観測を可能にすること、
%	\item 他の観測と共存・並行して突発天体を探査できるようにすること、
%	\item 突発天体の発見速報を受けた際に、すばやく観測モードを変更して追観測を行うこと。
%\end{itemize}
%SKAを長期間運用することを考えると、しばしば技術的な試験を行う必要性が出てくると思われる。
%そのときに例えば、一つあるいは複数のアンテナを用いて小規模な試験観測をできるような柔軟性があると、その後の観測が向上するだろう。
%また突発現象観測では第~\ref{transients.s2.fender}節で述べたように、ある観測と並行して別の観測も行えるような拡張性高いシステムと、緊急時に観測モードをすばやく再設定できるような柔軟性も必要である。
またそのようなシステム柔軟性の他に、観測後のデータ解析においては次の二つの要素が重要となる。

%\paragraph{要素 1: 一般市民による科学}
\paragraph{要求 1: Einstein@Home と同様のプロジェクト}
SKAを有効活用するには、その観測データが研究者だけでなく一般人にも開かれていることが重要である。
多くの人にSKAデータを解析してもらい、彼らの興味を最大限に引き出せるような環境を用意しなければならない。
この「一般市民による科学」には既に前例があり、重力波望遠鏡 LIGO や Arecibo 天文台の観測データを一般市民のコンピュータで解析してもらうプロジェクト、Einstein@Home\footnote{Einstein@Home: \url{http://www.einsteinathome.org/}} によって新しいパルサーが数多く発見されてきている。
同様のプロジェクトを SKA でも行うことで、SKAによる成果を最大化できるのみならず、人類全体による科学探求が実現するだろう。

%\paragraph{要素 2: 他の望遠鏡とのシナジー}
\paragraph{要求 2: Virtual Observatory への貢献}
SKAを有効に活用するには、前述の一般市民による科学と併せて「他の望遠鏡とのシナジー」を最大化することも重要である。
第~\ref{transients.s1}節で述べた (2) 自動追観測システムもそのシナジーのひとつであり、天体を連携して観測し多くの情報を得ることで、高いシナジー効果が生まれるだろう。
これは観測の最中に期待できるシナジーだが、観測が終わりデータだけが残っている状態でも、他の望遠鏡とのシナジーを図ることができる。
それを実現するのが Virtual Observatory\footnote{Virtual Observatory (VO; \url{http://www.ivoa.net/}) は観測データをデータベース化したり、扱いやすいデータ解析ツールを提供しているプロジェクトである。日本でも JVO (\url{http://jvo.nao.ac.jp/}) が活動し、一般市民もすばる望遠鏡などのデータにアクセスしやすくなっている。} (VO) であり、VO を利用することによって、複数の望遠鏡による多波長観測データを容易に比較できる。
したがって他の望遠鏡とのシナジーを飛躍的に高めるために、SKAはそのデータリソースをVOにつぎこむことが重要であり、それによって多くの科学成果を生むことができるだろう。


%%
\subsubsection{まとめ}
%%
%宇宙の現象のほとんどは既存の望遠鏡によって調べつくされている、と考える人もいるかもしれない。
%しかし実際にはそのようなことはない。
%なぜなら、それを調べようとして調べたものしか調べられていないからである。
%我々がまだ想像もしていないような未知の現象 (unknown unknowns) が宇宙にはあふれていると思われ、図~\ref{fig:transients.phasespace}の空白領域がその事を示唆している。
宇宙には、我々がまだ想像すらしていないような未知の現象 (unknown unknowns) があふれていると考えられ、図~\ref{fig:transients.phasespace}の空白領域がその事を示唆している。
アメリカの作家であり生化学者でもあるアイザック・アシモフが言ったとされている言葉に次のようなものがある。
\begin{quote}
%The most exciting phrase to hear in science, the one that heralds new discovery, is not ``Eureka!'' but ``That's funny...''\\
---科学において最も興奮する言葉、つまり新しい発見の前兆となる言葉は、「わかった!」ではなく「これは妙だな...」である。
\end{quote}
つまり known unknowns を解明したときの「わかった」という言葉よりも、unknown unknowns を発見したときの「妙だな」という言葉の方が、科学にとっては面白いということである。
SKAは、そのような未知の発見に備えて設計されなければならず、そのためには第~\ref{transients.s2.fender}節で述べたような柔軟な観測システムが不可欠である。
また、未知の発見に「備えておく」ことの重要性は、フランスの生化学者であるルイ・パスツールの言葉にも見ることができる。
\begin{quote}
%In the fields of observation chance favors only the prepared mind.\\
---チャンスはそれに備えている者だけにほほえむ。
\end{quote}
つまり、科学にとって最も面白い「これは妙だ」と言える未知を発見するには、その発見のチャンスが舞い降りることに備えておかなければならない。
その「備え」として、SKAには観測システムの「柔軟性」が不可欠である。
それとともに、一般市民が探査に参加できるようにすることや、他の望遠鏡とのシナジー効果を最大化することも重要となる。
それらを実現することで、SKAは長期にわたって科学成果を出し続けることができる。
