%%
%% Author: 前田啓一
%%
\subsection{系外超新星} \label{transients.s3.sn}

爆発直後の若い (一年程度以内の) 超新星における電波放射として支配的な機構は、超新星放出物質と星周物質の相互作用により発生する衝撃波と、そこで加速される相対論的電子からのシンクロトロン放射である \citep[e.g.,][]{1998ApJ...499..810C}。
これは現在の電波望遠鏡にとってはそれほど強い電波源ではなく、典型的な検出例では5~GHzにおいてピーク光度が$10^{27}~\text{erg}~\text{s}^{-1}~\text{Hz}^{-1}$ 程度の現象である \citep[e.g.,][]{2014arXiv1409.1827P}。
したがって、これまで電波観測されている数十例の系外超新星は、主に数十Mpc程度以内の距離で発生したものに限られており、非常にバイアスのかかった、限られたサンプルしか存在しない。

超新星の電波帯域における多様性は大きく、GRBに付随する極超新星 (SN 1998bwなど: \citealp{1998Natur.395..670G})、それと関連があると考えられている「相対論的」超新星 (SNe 2009bb, 2012ap: \citealp{2010Natur.463..513S})、あるいはIIn型超新星と呼ばれるタイプの一部 \citep[e.g.,][]{2012ApJ...755..110C} においては、その光度は$10^{29}~\text{erg}~\text{s}^{-1}~\text{Hz}^{-1}$ まで増光する。
タイムスケールも様々であり、5~GHz帯では典型的に数十日程度でピークを迎えるが、数年以上増光し続けるタイプのものも存在する。

観測されている電波放射の多様性は、超新星爆発により放出された物質の運動エネルギーおよび星周物質密度・分布の多様性を反映していると考えられ、電波放射の情報から星周物質密度などが導かれている \citep[e.g.,][]{2012ApJ...758...81M}。
一般に可視光観測からは星周物質の情報を直接導出することはできない。
超新星からの電波放射は、爆発直前の数十~数百年にわたる親星からの質量放出を反映するため、電波帯域における観測は、恒星進化論でも最大の謎の一つである大質量星の終末期の性質、超新星に至る直前の進化を知るためのユニークな手段である。
また、超新星残骸からの多波長放射により宇宙線加速機構の研究が進んでいるが、若い超新星においては衝撃波速度とダイナミクス、星周物質の密度が大きく異なる。
このため、宇宙線加速機構について超新星残骸を用いた研究とは別確度から切り込む新たな手段になると期待される。

SKAによる系外超新星研究は、主に可視で発見された超新星の即応追観測、SKA自体による電波サーベイの二つを並行して行うべきである。
現在世界中で様々な突発現象・超新星可視光サーベイが進行しており、SKAの時代には Large Synoptic Survey Telescope (LSST) の稼働が見込まれる。
日本においても木曽シュミット望遠鏡やすばる望遠鏡 Hyper Suprime-Cam (HSC) を用いた可視サーベイ計画が進展しつつある。
さらに、可視赤外大学間連携を通し可視・近赤外追観測の枠組みも整備され、恒星進化・超新星の理論研究を行うグループも多数存在しているなど、SKAの時代における即応追観測について存在感を発揮するための下地ができていると考えられる。
また、日本は歴史的に見ても電波観測におけるプレゼンスを発揮しており現在も大学間連携などを通し維持・発展しているのに加え、超新星電波放射の理論研究の下地もできつつある。
これらを総合し、可視--電波--理論を組み合わせた研究を進めるにあたり十分な下地ができていると言えるであろう。

\subsubsection{(1) 観測数の飛躍的増加によりもたらされる統計的理解}

超新星の性質の多様性を考えると、現在までの数十程度のサンプル数ではその全容を解明するためには全く不十分である。
SKAにおいては現在より少なくとも一桁良い観測精度を達成できるため、少なくとも数十倍の効率をもって、現在までの観測例とほぼ同等の質 (観測期間、S/N比など) のデータを取得できる。
たとえば年間一万個にのぼる発見が可能であるという見積もりも存在する \citep[e.g.,][]{2014arXiv1409.1827P}。
これにより、超新星の電波帯域における性質と可視光域における性質の関係が明らかになり、したがって爆発に至る進化 (電波) と超新星爆発の性質 (可視) の関係が明らかになるであろう。

\subsubsection{(2) まだ検出されていない電波放射の弱い超新星の研究}

従来の観測では、大半の超新星について電波放射は検出できていない。
つまり従来見つかっている超新星は、質量放出率の大きい系、爆発前に激しい質量放出をした系などが選択的に観測されたものであると考えられる。
また、既存の恒星進化理論では到底説明できないような、$0.1M_\odot~\text{year}^{-1}$以上の激しい質量放出が爆発直前に起こっている例も示唆されている \citep[e.g.,][]{2014ApJ...790L..16M}。
これらが特殊なのか一般的なのか明らかにするためには、より「弱い」電波放射をする超新星をとらえることが不可欠であり、これはSKAでなければ達成できない。

個々の問題についても、新たな電波放射検出が未解決問題解明のブレークスルーになり得る。
たとえば、Ia型超新星親星周辺の星周物質の有無・質量放出率は爆発に至る白色矮星の進化解明において最重要である \citep[e.g.,][]{2013FrPhy...8..116H}。
伴星が主系列星・巨星の場合 (single degenerate scenario: SD) には比較的濃い星周物質の存在が予測され、一方白色矮星二つの合体説 (double degenerate scenario: DD) からは希薄な周辺環境が予測される。
現在までの観測で、ごく近傍で発生した二つの超新星の星周密度について強い上限がついており、SD説に対し否定的な結果が得られている \citep{2014arXiv1409.1827P}。
同様の近傍Ia型超新星について弱い電波を実際に検出すること、さらに多くのサンプルについて強い上限値をつけることで、この議論に決着がつくであろう。
同様の議論はIa型超新星・重力崩壊型超新星の両方で見られる多様性の起源の解明のための大きな一歩となり得る。

\subsubsection{(3) 可視光では観測できない激しい星形成を起こしている銀河での超新星の検出と研究}

重力崩壊型超新星は大質量星の爆発であるため、その多くは激しい星形成を起こしている銀河で発生しているはずである。
しかし、luminous infrared galaxy (LIRG) や ultra luminous infrared galaxy (ULIRG)\footnote{光度が$10^{11}L_\odot$を超えるような赤外線で明るい銀河を LIRG とよび、$10^{12}L_\odot$を超えるものを ULIRG とよぶ。} といった星形成の激しい銀河は、一般的に可視光域では吸収が大きい銀河であり、そこで発生した超新星を可視光サーベイで発見することは困難である。
近年においては adaptive optics (AO) を用いた近赤外サーベイ、空間分解能の高い電波サーベイもなされているが、未だに数例の観測例が知られているだけである \citep[e.g.,][]{2007ApJ...659L...9M}。

超新星のイベントレートの測定は直接的な星形成率測定を与え、特に高赤方偏移に行くほど「隠された」星形成の割合は増えるため、イベントレートの解明は星形成史や初期質量関数の研究、超新星に至る恒星進化の研究にとって重要な要素である。
同時に、星形成の盛んな環境における超新星の性質が、これまで知られている超新星と同様であるかは自明ではない。
SKAにおける電波での広視野サーベイにより、このようなバイアスのかからない超新星のイベントレートや、様々な環境における超新星の性質・恒星進化、および初期質量関数の情報が得られるであろう。

\subsubsection{(4) 可視光観測とのシナジーによる近傍超新星の包括的研究}

これまでごく近傍の一握りの超新星について可視--電波の双方からの研究がなされているが、そのサンプル数の増大は重要である。
サンプル数が増えることによって、電波からわかる「親星の質量放出」と可視光からわかる「超新星の性質」との関係 (前述) や、様々な物理状況 (衝撃波速度、星周物質密度) における非熱的粒子加速 (後述) に加え、可視--電波観測のシナジーにより多くの新たな情報が得られる。
たとえば、電波による高い空間分解能の観測によって、ごく近傍の超新星については衝撃波伝搬の様子が直接撮像により求まる \citep[e.g.,][]{1995Sci...270.1475M}。また、爆発直後からの電波追観測により親星を衝撃波が突き抜けた直後 (ショックブレークアウト) の衝撃波速度を測定し、これから親星半径の情報を得ることができる \citep{2013ApJ...762L..24M}。
これらは最新の可視光観測手法と相補的であり、新たな研究分野を創出すると考えられる。

\subsubsection{(5) 強い衝撃波における相対論的電子加速機構、電子注入問題}

相対論的粒子加速機構は天体物理学における最重要未解明問題の一つである。
加速陽子の最大エネルギーが近年明らかになりつつあるが、それと同様に大きな問題として低エネルギー電子注入問題がある。
星周物質中を衝撃波が伝搬する若い超新星においては、物質密度したがって増幅磁場密度が大きいことがわかっており、GHz帯での観測によりMeV--GeV域の電子からの放射をとらえることができる \citep{2013ApJ...762L..24M}。
さまざまな星周物質密度をもつ異なる超新星の観測を包括的に行うことで、MeV--GeV域での加速電子スペクトルを得ることができるほか、ALMAなどによる高周波観測を組み合わせることで、個々の超新星における低エネルギー非熱的電子スペクトルを得ることができるであろう。
これは、電子注入問題を解明する上でのブレークスルーになり得る。
また、様々な環境で発生し様々な衝撃波物理状況を持つ様々な超新星を観測することにより、電子加速効率などを統計的に調べることが可能になり、これをもって粒子加速機構に必要・重要な条件を探ることができるであろう。



