\subsection{Fast Radio Bursts (FRBs)} \label{transients.s1.frb}
銀河系外からの電波パルスは、1930年代に電波天文学が拓かれて以降近年に至るまで、発見されてこなかった。
つまり\Tabref{tb:frail-classification}の左上の欄は、近年まで空白のままであった。
しかしついに2007年、銀河系外起源と思われる電波パルスがオーストラリアParkes 64~m 電波望遠鏡によって発見された \citep{2007Sci...318..777L}。
そのパルスは幅数ミリ秒と短寿命かつ大強度であり、さらに発見された位置が高銀緯にも関わらず分散測度 (dispersion measure; DM) が極めて大きいことから、その起源は銀河系外の高エネルギー天体だと考えられた\footnote{
最初に発見されたFRB~010724 (Lorimer burst) は、パルス幅が $W=5~\text{ms}$、フラックス密度が $S_{1.4~\text{GHz}}=30~\text{Jy}$、また発見された位置が銀緯$b = -41.8\arcdeg$、パルスの分散測度は$\text{DM} \sim 375~\text{pc}~\text{cm}^{-3}$であった。
それから推定される地球からの距離はおよそ500~Mpcと考えられ、また遠くとも1~Gpc以内の天体だと考えられている。
}。
さらにパルス放射はその1発限りであり、それ以降同じ場所からの電波放射は確認されていない。

このような現象は従来報告されておらずその信憑性が疑われたこともあり、しばらくは発見者の名前をとってLorimer burst と呼ばれていたが、同様のパルスが \citet{2011MNRAS.415.3065K} および \citet{2013Sci...341...53T} によって新たに5例発見され、この銀河系外からの電波パルスは fast radio bursts (FRBs) と呼ばれるに至った。
このFRBの起源はわかっておらず、そもそもPerytonと命名されている地球起源の謎のノイズ \citep{2011ApJ...727...18B} との区別も明確ではない。
しかしその後、オーストラリアから遠く離れたプエルトリコのArecibo 300~m 電波望遠鏡によって同様のFRBが1例発見され \citep{2014ApJ...790..101S}、FRBが宇宙起源の天体現象であることの信頼性は非常に高まっている。
起源天体のモデルとしては多くの説が唱えられており、
銀河系内からの放射では説明しにくいことが様々な研究で示されている \citep[e.g.,][]{2014ApJ...797...70K}。
%ただし一方で系内起源を主張する研究もあり \citep[e.g.,][]{2014MNRAS.439L..46L}。
%例えば中性子星連星の合体に伴う電波放射モデルが提案されており (\citealt{2013PASJ...65L..12T})、合体に伴う重力波放射との一致性が確認されれば宇宙物理学に大きな影響を与えるだろう。

