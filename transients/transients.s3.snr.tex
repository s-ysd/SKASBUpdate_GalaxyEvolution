%%
%% Author: Shiu-Hang Lee, 長瀧重博
%%
\subsection{超新星残骸における粒子加速の現場、および星間物質と磁場の相互作用} \label{transients.s3.snr}

質量の大きな星々はその最期に超新星爆発とよばれる大爆発を起こすが、その後に何も残さない訳ではなく多様な物理現象がその後も続く。
多くの場合、超音速で膨張する星の残骸 (イジェクター) が様々な星間物質、あるいは星が「生前」放出した星風に衝突することによって衝撃波を形成する。
また衝撃波によってガスが高温に加熱され、陽子・電子などの帯電粒子が高エネルギーに加速されるなど非熱的な現象をも起こし、その結果電波・赤外線・可視光からX線・ガンマ線に至るまで多波長で見られる星雲状の天体を残す。
その何光年にも広がる天体は超新星残骸と呼ばれる。

超新星残骸は銀河の中の物理現象において様々な重要な役割を果たしていると考えられている。
例えば銀河系内宇宙線の生成、鉄など生命に不可欠な重元素の放出、星の形成、星間磁場の増幅などは、超新星残骸によって引き起こされるものである。
しかし、これまでに数多くの観測、理論モデリング、シミュレーションがなされてきたにもかかわらず、残骸中の物理は未だ完全には解明されておらず、実に複雑な天体であることが分かっている。
その問題が未解決であることの原因は、超新星残骸が年齢、親星の構造・組成、周囲の環境などの違いによって著しい多様性を示すこと、また既存の観測装置の性能限界とサンプル数の不足のためである。
そこでSKAのような、従来の観測装置の性能を遥かに超える次世代望遠鏡が、超新星残骸の理解のために非常に期待されている。
SKAを用いた科学や、そのためのSKAへの要求としては、以下のようなものが挙げられる。

\paragraph{(1) サンプル数の増大}
今まで発見された銀河系内の超新星残骸の数は僅か300に満たないが \citep{2014BASI...42...47G}、超新星の発生率を考えれば \cite[e.g.,][]{2006Natur.439...45D}、この数字は極端に少ない。
このことはつまり、まだ発見されていない超新星残骸が数多く残っているということであり、SKAの空間分解能と高感度があれば、これまでは暗すぎて観測できなかった残骸や、さらには系外天体の残骸についても多数発見出来ると考えられる。
また近傍銀河 (大マゼラン星雲、M33銀河など) の超新星残骸を探査することにより、若い残骸から古い残骸まで、超新星残骸の全体像を統計的に描けるようになる。
近傍銀河内の超新星残骸観測により、吸収のために観測しにくい天の川銀河内の残骸の真の分布も間接的に予想できるかもしれない。
更に他波長の観測装置 (Astro-H、CTA、ALMAなど) と緊密に連携し、多くの残骸を幅広いエネルギー域で調べることも可能となるだろう。

\paragraph{(2) 残骸の真の構造}
電波連続波で見る残骸はほとんどシェル状だが、その内部構造は複雑であり、特に衝撃波の周りではフィラメント状の構造が見つかる場合がある \citep[][]{1997ApJ...491..816R,2007A&A...471..537C,Reynolds2011}。
SKAを駆使することで、今まで感度不足のため見えなかった細かい構造を定量化し、モデリングを通して超新星残骸の真の三次元構造を明らかにできるだろう。
特にX線など他の波長との相関がより明確にわかるようになるため、超新星残骸中の非熱電子と磁場強度の分布を同時に解明できるようになり、無衡突衝撃波における粒子加速と磁場増幅の物理への理解を更に深めることにつながる。

\paragraph{(3) 粒子加速と磁場}
超新星残骸の電波連続波は主に加速された非熱電子によるシンクロトロン放射である。
SKAは広い波長帯をカバーしており、場所ごとの非熱電子のベキを精度よく測定できるようになると期待される。
また高感度で偏波観測を行えば、偏波率の小さい残骸に対しても、残骸内外の磁場の方向を推定することができる。
磁場の方向は衝撃波による粒子加速に大きく寄与するため \citep[][]{1996ApJ...473.1029E,2013AJ....145..104R,2014ApJ...783...91C}、SKAは高感度な偏波観測を行えるようデザインされなければならない。

\paragraph{(4) 星間物質との相互作用}
超新星残骸の多くは周りの星間物質 (星風、HI・分子雲など) と相互に強い影響を与え合い、明るい電波を放つ \citep{2000ApJ...538..203B,2012ApJ...746...82F}。
例えば、最近フェルミ衛星 \citep[e.g.,][]{2013Sci...339..807A} の観測によって、ガンマ線で明るくかつ爆発から1万年程度経った古い残骸は、そのほとんどが分子雲とぶつかっており、電波でも明るく観測されることがわかっている。
その放射機構・粒子加速機構はまだ謎のままだが、SKAによってこのような残骸を精度よくイメージングし、他の波長の観測データと組み合わせ、更にセルフコンシステントな理論モデル \cite[e.g.,][]{0004-637X-750-2-156} で解釈することにより、この謎は解明されると考えられる。
このとき他波長データとの比較のため、SKAはその資源を\Secref{transients.s2.wilkinson}で述べたVirtual Observatoryにささげることが重要である。


