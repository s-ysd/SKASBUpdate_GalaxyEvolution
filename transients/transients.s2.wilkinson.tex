\subsection{未知の未知} \label{transients.s2.wilkinson}
アメリカ合衆国国防長官だったラムズフェルド氏の言葉 \citep{Rumsfeld} を借りると、物事には三種類あり、それらは
\begin{description}
	\setlength{\leftskip}{1cm}
	\item[Known knowns:] 我々が既に知っている事実、
	\item[Known unknowns:] 我々が知らないことを自覚している未知、
	\item[Unknown unknowns:] 我々が知らないことを自覚すらしていない未知、
\end{description}
というものである。
このことは、宇宙科学の分野においてもまったく同様である。
そこで本節では、unknown unknowns つまり未知の未知に相当する、予想すらできていない未知の天体現象の探査 (exploration of unknowns; EoU) について述べる。
とりわけ未知を発見するために持っておくべき哲学と、その哲学に基づいたSKAへの要求仕様を紹介する。

%%
\subsubsection{未知追究の哲学}
%%
宇宙科学における known knowns は既に研究されてきていることであるから、もしそれについて知りたければ文献を読めばよい。
一方 known unknowns について知りたければ、何を知るべきか把握していることなので、観測提案書を書いて観測を実施すればよい。
その観測が成功すれば、known unknowns は known knowns に変わり、宇宙科学は一歩前進することになるだろう。
しかしながら unknown unknowns については、我々はその存在にすら気付いておらず「知らないということすら知らない」。
そのため、どんな研究をしてどんな観測をすれば良いのか、検討するどころか見当すらつかない。
しかし過去の偉大な科学的発見が、そのような unknown unknowns を偶然発見したことに端を発しているのも事実である。

ゆえにSKAが長期に渡って科学成果を出し続けるためには、そのような unknown unknowns を発見できる能力を備えておかなくてはならない。
SKAは、前章までに述べてきたような「現在」未解決とされている問題に対して注力し、その解決を目標とする。
%それらの問題の解決は非常に重要であり、その実現によって宇宙科学は大きく前進することだろう。
しかしSKAが完成し隆盛を誇っているだろう2025年以降には、現在未解決のそれらの問題は既に解決されているに違いない。
その時SKAによる興奮はどこにあるかと言えば、それは解決済みの古い問題にはなく、新しい観測方式によって浮上するだろう新しい疑問、つまり unknown unknowns にある。
%人間はこの宇宙や構成物を第一原理から創り出せるほど、創造力も想像力もない。
それゆえ天文学では、我々が想像すらしていない「未知の何か」を見つけるための「備え」が重要であり、その備えに裏打ちされた「発見」が重要となる。

%%
\subsubsection{未知の探査 (EoU) のための要求}
%%
電波天文学における unknown unknowns がどのような天体現象なのかはもちろん予測できないが、「起源のわからない未知の突発現象」が既に発見されてきている。
%前述のとおり電波帯域における突発現象の中には、それを放射した実体がどのような天体なのか、あるいはどのような放射機構によって突発的な電波放射がなされたのか、わかっていないものが少なくない。
%第~\ref{transients.s1}節で述べたとおり、電波帯域における突発現象の中にはそれを放射した実体や放射機構がわかっていないものが多い。
%そのような未知の突発現象は、MHz から GHz という低周波帯域で、集光面積の大きな望遠鏡によって発見されていることが多い。
しかしそれらを発見した望遠鏡は、そのような未知の現象を捉えようとして設計されたわけではなかったし、何らかの理論をもとにして突発天体を探査したわけでもなかった。
ところが蓋を開けてみれば、未知の突発現象が次々と発見され、新たな科学が生まれている。
そしてこれからも、まだ予想もされていないような変動現象や突発現象が次々に見つかるに違いない。
それを効率的に発見できるのが、SKAである。

SKAは高い感度と広い視野を持っており、未知の現象を数多く発見できる潜在能力がある。
ただし潜在能力があったとしても、それを有効に活用できなければ未知の探査は進まないだろう。
SKAがその潜在能力を解放し未知の探査を効率的に行うためには、第~\ref{transients.s1}節で述べたような (1) 共存システムや (2) 自動追観測システムを活用できるような「柔軟性」が必要である。
%SKAがその潜在能力を解放し未知の探査を効率的に行うためには、次のようなシステムの柔軟性と拡張性が必要となる:
%\begin{itemize}
%	\item 小規模な試験観測を可能にすること、
%	\item 他の観測と共存・並行して突発天体を探査できるようにすること、
%	\item 突発天体の発見速報を受けた際に、すばやく観測モードを変更して追観測を行うこと。
%\end{itemize}
%SKAを長期間運用することを考えると、しばしば技術的な試験を行う必要性が出てくると思われる。
%そのときに例えば、一つあるいは複数のアンテナを用いて小規模な試験観測をできるような柔軟性があると、その後の観測が向上するだろう。
%また突発現象観測では第~\ref{transients.s2.fender}節で述べたように、ある観測と並行して別の観測も行えるような拡張性高いシステムと、緊急時に観測モードをすばやく再設定できるような柔軟性も必要である。
またそのようなシステム柔軟性の他に、観測後のデータ解析においては次の二つの要素が重要となる。

%\paragraph{要素 1: 一般市民による科学}
\paragraph{要求 1: Einstein@Home と同様のプロジェクト}
SKAを有効活用するには、その観測データが研究者だけでなく一般人にも開かれていることが重要である。
多くの人にSKAデータを解析してもらい、彼らの興味を最大限に引き出せるような環境を用意しなければならない。
この「一般市民による科学」には既に前例があり、重力波望遠鏡 LIGO や Arecibo 天文台の観測データを一般市民のコンピュータで解析してもらうプロジェクト、Einstein@Home\footnote{Einstein@Home: \url{http://www.einsteinathome.org/}} によって新しいパルサーが数多く発見されてきている。
同様のプロジェクトを SKA でも行うことで、SKAによる成果を最大化できるのみならず、人類全体による科学探求が実現するだろう。

%\paragraph{要素 2: 他の望遠鏡とのシナジー}
\paragraph{要求 2: Virtual Observatory への貢献}
SKAを有効に活用するには、前述の一般市民による科学と併せて「他の望遠鏡とのシナジー」を最大化することも重要である。
第~\ref{transients.s1}節で述べた (2) 自動追観測システムもそのシナジーのひとつであり、天体を連携して観測し多くの情報を得ることで、高いシナジー効果が生まれるだろう。
これは観測の最中に期待できるシナジーだが、観測が終わりデータだけが残っている状態でも、他の望遠鏡とのシナジーを図ることができる。
それを実現するのが Virtual Observatory\footnote{Virtual Observatory (VO; \url{http://www.ivoa.net/}) は観測データをデータベース化したり、扱いやすいデータ解析ツールを提供しているプロジェクトである。日本でも JVO (\url{http://jvo.nao.ac.jp/}) が活動し、一般市民もすばる望遠鏡などのデータにアクセスしやすくなっている。} (VO) であり、VO を利用することによって、複数の望遠鏡による多波長観測データを容易に比較できる。
したがって他の望遠鏡とのシナジーを飛躍的に高めるために、SKAはそのデータリソースをVOにつぎこむことが重要であり、それによって多くの科学成果を生むことができるだろう。


%%
\subsubsection{まとめ}
%%
%宇宙の現象のほとんどは既存の望遠鏡によって調べつくされている、と考える人もいるかもしれない。
%しかし実際にはそのようなことはない。
%なぜなら、それを調べようとして調べたものしか調べられていないからである。
%我々がまだ想像もしていないような未知の現象 (unknown unknowns) が宇宙にはあふれていると思われ、図~\ref{fig:transients.phasespace}の空白領域がその事を示唆している。
宇宙には、我々がまだ想像すらしていないような未知の現象 (unknown unknowns) があふれていると考えられ、図~\ref{fig:transients.phasespace}の空白領域がその事を示唆している。
アメリカの作家であり生化学者でもあるアイザック・アシモフが言ったとされている言葉に次のようなものがある。
\begin{quote}
%The most exciting phrase to hear in science, the one that heralds new discovery, is not ``Eureka!'' but ``That's funny...''\\
---科学において最も興奮する言葉、つまり新しい発見の前兆となる言葉は、「わかった!」ではなく「これは妙だな...」である。
\end{quote}
つまり known unknowns を解明したときの「わかった」という言葉よりも、unknown unknowns を発見したときの「妙だな」という言葉の方が、科学にとっては面白いということである。
SKAは、そのような未知の発見に備えて設計されなければならず、そのためには第~\ref{transients.s2.fender}節で述べたような柔軟な観測システムが不可欠である。
また、未知の発見に「備えておく」ことの重要性は、フランスの生化学者であるルイ・パスツールの言葉にも見ることができる。
\begin{quote}
%In the fields of observation chance favors only the prepared mind.\\
---チャンスはそれに備えている者だけにほほえむ。
\end{quote}
つまり、科学にとって最も面白い「これは妙だ」と言える未知を発見するには、その発見のチャンスが舞い降りることに備えておかなければならない。
その「備え」として、SKAには観測システムの「柔軟性」が不可欠である。
それとともに、一般市民が探査に参加できるようにすることや、他の望遠鏡とのシナジー効果を最大化することも重要となる。
それらを実現することで、SKAは長期にわたって科学成果を出し続けることができる。
