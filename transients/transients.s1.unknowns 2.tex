%%
%% Unknowns
%%
\subsection{未知の突発天体} \label{transients.s1.unknowns}
本節では電波帯域でのみ観測され、他の波長帯域やあるいは電波帯域ですら対応天体が見つかっていないような未知の突発天体について紹介する。
併せて、既知の天体と同定はできているものの、その放射過程が未解明な天体についても述べる。
その紹介にあたっては観測方法に着目した\Tabref{tb:frail-classification}の分類法に従い、その天体が銀河系外か系内にあるか、短時間変動か長時間変動か、という4種類に分けて述べる。

%%
%% ss1
%%
\subsubsection{種類1: 銀河系内の短時間変動天体}
銀河系内の天体に起源を持ち、変動の継続時間が数秒以下のパルスを発する天体として代表的なものはパルサー、つまり自転する中性子星である。
それらパルサーの中でも、Crab nanoshots や RRATs といった特殊なパルス放射現象が報告されおり (次項参照)、それらの現象は突発天体研究においても興味深い研究対象である。
また銀河系内でパルス状の電波放射をする天体としては中性子星のみが知られており、それ以外の系内天体で電波パルスは観測されていないが、今後、中性子星起源とは考えられないような電波パルスが観測されることもあるかもしれない。
そのような未知の突発現象観測には、従来の望遠鏡を越える感度と高い時間分解能、そしてデータ解析コンピュータの計算速度が要求される。

%%
\paragraph{Crab Nanoshots}
電波パルサーの中には、通常のパルス放射とは異なる放射をするパルサーがあることが、近年明らかになってきている。
Crabパルサーから放たれるパルスは、そのパルス幅が 3~ms、パルス周期が 33~msであり、そのフラックス密度は平均して 14~mJyである。
一方で時折、大強度のジャイアントパルスを放つことがあり、そのフラックス密度は 1~MJy に達することもある。
このジャイアントパルスの中で、パルス幅が通常よりも極端に短くなる現象がArecibo 電波望遠鏡を用いた周波数 9.25~GHz の観測によって明らかとなった。
そのパルス幅は観測の時間分解能 0.4~ns より短く、フラックス密度は 2~MJy という極めて短寿命で大強度のパルスであり、nanoshots と呼ばれた \citep{2003Natur.422..141H,2007ApJ...670..693H}。
そのパルスの放射源の大きさは 12~cm、輝度温度は$10^{41}~\text{K}$と見積もられ、高エネルギーかつコンパクトな天体現象である。

発見者らはこのnanoshotsを説明できる唯一のモデルとして \citet{1998ApJ...506..341W} によって提示されたプラズマ乱流モデルを挙げているが、その放射機構は必ずしも解明されていない。
また同様にパルス幅が短くなる現象は他のパルサーでは確認されておらず、現状ではCrabパルサーに特有の現象である。
さらにはジャイアントパルスでのみ起こるのか、本質的には通常のパルスでも起こっているが感度不足で検出できていないだけなのかという点もわかっていない。
SKAを用いた高感度観測によってnanoshotsについてより詳しい知見が得られれば、パルサー磁気圏の研究が大きく進むと期待できる。

%%
\paragraph{RRATs}
通常のパルサーとは異なるパルス放射天体としては、Parkes 64~m 電波望遠鏡で発見された rotating radio transients (RRATs) が挙げられる \citep{2006Natur.439..817M}。
このRRATsはパルサーと同様に強い磁場を持つ中性子星が起源だと考えられているが、パルサーの周期的なパルス放射とは異なり、その放射は散発的で放射機構は明らかとなっていない。
SKAを用いてRRATsを高感度で観測すれば、その散発性についてより詳しい議論が可能になるだろう。


%%
%% ss2
%%
\subsubsection{種類2: 銀河系外の短時間変動天体}
銀河系外からの電波パルスは、\Secref{transients.s1.frb}で述べたFRBのみである\footnote{FRBは\Secref{transients.s1.frb}で述べたように銀河系内を起源とする説もあり、必ずしも系外天体として見解が一致しているわけではない。}。
FRBは従来予想もされていなかった現象であり、発見当初はそれほど大きな注目を集めていなかった。
しかしその後アーカイブデータの解析によって続々と発見されたため、現在は大きな注目を集め、SKAによる重要なサイエンスの一部を担うに至っている。
FRBのような未知の天体は、まだ宇宙に多く眠っていると考えられ、それらを発見するには柔軟な観測システムが必要となり、SKAはそれを実装しなければならない (\Secref{transients.s2.fender})。

%%
%% ss3
%%
\subsubsection{種類3: 銀河系内の長時間変動天体}
銀河系内を起源とした数秒以上の長時間変動を示す天体は数多くあり、例えば\Tabref{tb:frail-classification}に挙げた恒星フレアやメーザーバースト、X線連星における電波バーストなど多種多様である。
それらのほとんどは対応天体が見つかりやすく、起源がわかっているものが多い。

一方で対応天体が同定されていないものもあり、例えば\citet{2005Natur.434...50H} が銀河中心近傍に発見した周期的なバースト現象が挙げられる。
このバーストはフラックス密度が 1~Jy、1回のバーストの幅が 10 分でそのバーストが 77 分周期に5回現れるという特異的な突発現象であり、VLA を用いた観測によって発見された。
この天体は GCRT J1745-3009 と名付けられ\footnote{接頭辞 GCRT は Galactic Center Radio Transient の頭文字をとったものである。}、対応天体は見つかっておらず放射機構も不明である。
しかしその後の追観測によって同じ場所から同様のバーストが観測され、\citet{2010ApJ...712L...5R} によって偏波情報が明らかとなった。

その情報などから起源についていくつかのモデルが発案されており、その中には、一時的にパルス放射が消える nulling pulsar (\citealp{2005Natur.434...28K})、中性子星連星 (\citealp{2005ApJ...628L..49T})、白色矮星 (\citealp{2005ApJ...631L.143Z})、歳差運動するパルサー (\citealp{2006MNRAS.365L..16Z})、恒星フレア (\citealp{2010ApJ...712L...5R}) などがある。
以上のように起源について多くのモデルは考えられているが、それらについて確証は得られていない。
このような数分スケールのバースト現象の探査は大規模には行われておらず、SKAにおいてもデータ解析システムに対して同様のバーストを効率的に探査する機能を要求する必要がある。

%%
%% ss4
%%
\subsubsection{種類4: 銀河系外の長時間変動天体}
銀河系外における長時間変動を示す天体のうち、対応天体が見つかりにくく未知天体と認識されるものとして代表的なのは、orphan GRB afterglow (\Secref{transients.s1.grb}) や TDE (\Secref{transients.s1.tde}) であろう。
GRB残光や超新星爆発に代表される突発現象は、電波帯域では増光は急激だが減光は緩やかで、その継続時間は数か月以上に及ぶものも多い。
一方、変動の継続時間が数か月以下で、起源が未知の突発天体がいくつか報告されている。

例えばVLA のアーカイブデータからは4つの候補天体が発見されており、特にRT~19920826と命名されたものについては検出状態も良好であった \citep{2007ApJ...666..346B,2012ApJ...747...70F}。%
同様の突発天体は早稲田大学の那須パルサー観測所においても発見され、WJN~J1443+3439と命名された天体があるが、その起源については恒星フレアやAGNフレアが示唆されているものの、必ずしもわかっていない \citep{2007ApJ...657L..37N,2014ApJ...781...10A}。
そのような未知の突発天体を探査し、\Figref{fig:transients.phasespace}の空白領域を埋めるには、広い視野を高感度で探査し、さらに様々な時間分解能でデータ処理するための解析システムが必要になる。

%%\subsubsection{LOFAR Transients を加筆}