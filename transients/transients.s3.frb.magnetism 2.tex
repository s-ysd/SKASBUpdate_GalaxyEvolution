\subsection{FRBの偏波の可能性と宇宙磁場} \label{transients.s3.frb.magnetism}

今現在までに、FRBに有意な直線偏波は検出されていない\footnote{円偏波は1例観測された \citep{2015MNRAS.447..246P}。}。それがFRB現象に普遍的な現象なのか、または偶然なのか、その判断をするにはまだ発見数はあまりにも少ない。もしFRBに直線偏波が含まれるなら、ファラデー効果の分散測度 (DM) に加えてファラデー回転測度 (RM) も計測できる可能性がある。そうなった場合、密度重みつき視線平均磁場強度$\langle B_\parallel \rangle $をDMとRMの定義から
\begin{equation}
\langle B_\parallel \rangle =\frac{{\it RM}}{{\it DM}}
\end{equation}
という式で簡単に求めることができる。FRBが系外起源であるならば、この磁場強度の推定には銀河間磁場の情報を含む。電波銀河など系外偏波源を使ってRMを測る場合はDMは測らない (測れない) ため、このように単一観測から磁場強度まで推定できるのは画期的なことである。一つの視線だけでは系内と系外の寄与を切り分けるのは難しいかもしれないが、沢山の観測で統計を高めることで、銀経・銀緯に依存しないが赤方偏移に依存するような「超過成分」として銀河間磁場の寄与が議論できるかもしれない。SKAのサーベイ能力は、そのような統計的な議論をはじめて可能にするだろう。具体的で確実な調査の方法論を理論的に検討していくことも不可欠であるが、日本は銀河間磁場の調査で世界をリードしているので\citep{2014PASJ...66...65A,2014ApJ...790..123A}、日本のSKAサイエンスの特色の一つとできるだろう。FRBが銀河間磁場を探る新しい方法となるかもしれない。