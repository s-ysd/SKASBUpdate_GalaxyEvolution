\subsection{ブラックホールによる星の潮汐崩壊} \label{transients.s1.tde}
超巨大ブラックホール (supermassive black hole; SMBH) のまわりを回る星々は、そのブラックホールから強い潮汐力を受け、もしその潮汐力の大きさが星の重力を越えると、その星は形状を保つことができず崩壊してしまう。
これを潮汐崩壊現象 (tidal disruption event; TDE) とよび、その際、まれにジェットを伴う爆発を起こし急激に光度を増す。

そのようなジェットを伴う潮汐崩壊現象は、特異的なガンマ線バーストとしてSwift衛星によって初めて観測され Swift J1644+57 (GRB 110328A) と命名された。
発見当初は起源がわからなかったが、その後、X線、赤外線、電波の追観測によって残光が観測され \citep{2011Sci...333..199L}、潮汐崩壊現象と考えられるにいたった \citep{2011Sci...333..203B}。
SKAはこの潮汐崩壊現象を数多く発見できると考えられ、銀河中心ブラックホールの物理に迫ることができるだろう。

