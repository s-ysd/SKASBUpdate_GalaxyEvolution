%% Author: 榎戸 輝揚
\subsection{マグネター磁気圏の解明} \label{transients.s3.magnetars}
X線パルサーの中には、自転周期 2--12~s という遅い自転と、その自転周期が毎秒 $10^{-12}$--$10^{-10}~\text{s}$で遅くなるという極めて早い減速率を示し、そこから推定される磁場の強さが $10^{14}$--$10^{15}~\text{G}$ にも達するものが報告されている \citep{2008A&ARv..15..225M}。
さらに、観測されるX線光度が中性子星の自転では説明できないこともあり、それらのパルサーは内部に蓄えた磁場を解放して輝く超強磁場の天体と考えられるようになり、マグネターと呼ばれるようになった \citep{1995MNRAS.275..255T}。
このマグネターは、これまで銀河系内に 2000 個以上見つかってきた通常の回転駆動型パルサーとは異なる種族として、超新星爆発の機構や磁場の起源、強磁場の物理などの観点からも注目を集めている。

Swift衛星によって、突発的にX線で増光する銀河系内のマグネターも数多く見つかるようになり、すでに 30 個程度が知られている。
この発見をもたらした X 線アウトバースト現象では、$k_\text{B}T \sim 0.3~\text{keV}$ にピークをもつ星表面の熱的放射が増光するとともに、硬X線にピークをもつショートバーストを頻発し 100~keV 近くに向かって伸びる硬X線のべき放射も検出されるようになってきた \citep{2010ApJ...715..665E}。
こういった突発現象は通常の電波パルサーと異なるマグネターの特徴で、多波長でのフォローアップが急速に進展している。

通常の電波パルサーと異なり、定常的に明るく輝くマグネターでは、多くの場合電波のパルス放射が検出されていない。
たとえばGreen Bank 望遠鏡を用いた 1950~MHzの電波観測では、マグネター候補7天体で電波パルスは観測されなかった \citep{2012ApJ...744...97L}。
しかし一方で、X線アウトバーストを生じた際には電波が放射されるという例が報告されるようになってきた。
2003年に発見された XTE~J1810-197 \citep{2006Natur.442..892C} や2007年に突発増光した 1E~1547.0-5408 (PSR~J1550-5418; \citealt{2008ApJ...679..681C}) では、GHz 帯域でフラットなスペクトルを示し、100\% に近い直線偏光を示すなど、共通点をもった電波パルスが報告された。
さらに、電波観測によって最初に発見されたマグネター PSR~J1622-4950 \citep{2010ApJ...721L..33L}や、2013 年に銀河中心で見つかったマグネター SGR~1745-29 (PSR~J1745-2900; \citealt{2013ApJ...770L..23M}, \citealt{2014ApJ...780L...3S}) など、電波帯域でもマグネターの観測例が蓄積されつつある。

マグネターはX 線で卓越した放射を示すため、これまでは主に高エネルギー帯域の観測が主だったが、トランジェント型のマグネターの電波観測が新しい観測の窓とし
て開かれつつある。
また、定常的に明るい天体からも微弱な電波パルスが検出できるようになることも期待される。
従来、電波パルサーの多波長に渡るパルス研究がパルサー磁気圏の理解を進ませたように、SKA による電波観測を可視光やX線と組み合わせた多波長の突発現象観測が、マグネターの磁気圏の解明の伴になると期待できる。