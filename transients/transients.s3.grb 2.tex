%%
%% Author: 長瀧 重博
%%
\subsection{電波によるガンマ線バーストの即応追観測} \label{transients.s3.grb}
宇宙最大の爆発とされるガンマ線バースト (gamma-ray burst; GRB) はまず数100~keV程度のソフトガンマ線で明るく輝き (1--10秒程度続く即時放射と呼ばれる)、次いでX線、紫外・可視・赤外光、電波など多波長で輝く「残光」と呼ばれるフェーズに移行していく。
現在GRB検出の主力観測機器はNASAのSwift衛星であり、ソフトガンマ線検出器に加えX線検出器、紫外・可視光検出器を搭載している。
視野についてはソフトガンマ線検出器が優れているのに対し、位置決定精度についてはX線検出器、紫外・可視光検出器が優れている。
従って即時放射をまず広い視野を持つソフトガンマ線検出器で同定し、直ちにその視野内をX線、紫外・可視光検出器でフォローアップすることで残光を検出、GRBの正確な位置情報を地上望遠鏡等に速報するシステムが構築されている (\Secref{transients.s1.grb})。
このシステムにより、Swift衛星によって秒角程度に絞られた位置情報を基に、すばる等の地上望遠鏡は深いフォローアップを行い、詳細な残光観測・ホスト銀河観測等を行うことが現在可能となっている。

\subsubsection{SKAへの期待と要求}
SKAの時代にSwift衛星自体が運用されているかどうかは定かではないが、類似のGRB衛星計画にはSVOM計画 (中国--フランス: \citealt{2011CRPhy..12..298P})、HiZ-GUNDAM計画 (日本: \citealt{2014SPIE.9144E..2SY}) などがあり同様の速報体制は維持されているものと考えられる。
従ってGRBはトランジェントな現象であるが、SKA自体がサーベイする必要性はそれ程高くない。
むしろ速報を受けた際に素早くフォローアップする用意をSKAは行っておくべきである。
またこれまでに観測されているGRBの典型的な電波残光のスペクトルから、GRB電波残光はGHz以上の周波数帯で明るいことが知られている。
以上の状況から、SKAによるGRB観測はたとえ視野が狭くても大きな問題ではなく、GHz以上の観測波長帯で感度の高い\skamid{}が有効であると考えられる \citep{2015arXiv150104629B}。

\skamid{}の感度を考慮すると、およそ年間数百のGRB電波残光が検出できると予想される。
この数は1997年に初めて残光が検出されて以降、観測された電波残光の総数とほぼ同程度であり \citep{2012ApJ...746..156C}、SKAは電波残光の観測例を飛躍的に増大させると期待できる。
またこれまでの統計では、発見されたGRBの約1/3の割合のみで電波残光が検出されているが \citep{2012ApJ...746..156C}、SKAによる深い観測によって``弱い''電波残光が検出されることも期待され、GRB残光の物理状況の理解がより進むと期待される。
また比較的近傍の電波残光については放射領域の固有運動の検出が期待でき \citep{2004ApJ...609L...1T}、GRBのダイナミクスに重要な情報を与えると期待できる。
遠方の電波残光については特に初期のフェーズにおいてシンチレーションによる光度変動が予想され \citep{1997Natur.389..261F}、GRBジェットの口径に強い制限を与えることもできる。
更にGRBジェットが地球を向いていない場合にも、後期にはジェットの減速に伴い口径が広がり、ついには地球で残光が検出されるという親なしGRB残光 (orphan afterglow, \Secref{transients.s1.grb}) の初検出も\skasur{}において期待できる \citep{2015arXiv150104629B}。
また起源のわかっていない short GRBの電波残光は現在まで30例強発見されているのみで、統計的に充分でない。
しかしこの限られたサンプルの中では long GRBに比べて一桁、あるいはそれ以上電波残光の強度が低い傾向が見られるという報告もある \citep{2012ApJ...746..156C}。
SKAによる高感度観測により、short GRB電波残光の特性がより明確に理解でき、short GRBの正体に迫る手掛かりにもなると期待される。

\subsubsection{日本の独自性とまとめ}
以上のようにSKAのGRBに対する科学的貢献は期待が非常に高い。
そしてSKAがその期待に応えるためには、GRBの発見速報を受けた際に、素早くそれをフォローアップできるような運用体制を\skamid{1}に課さなければならない。
また日本 (SKA-Japan) においては、日本のGRBミッション・HiZ-GUNDAMとの連携で大きな特色を打ち出すことが期待できる。
HiZ-GUNDAMは高赤方偏移 (high-z) した遠方GRBの検出を主眼においており、宇宙再イオン化期の宇宙の物理状況を解き明かすことにも貢献できるミッションである。
すなわちSKAの21~cm線観測による宇宙再電離期観測とは別のアプローチとして、SKAとHiZ-GUNDAMとの連携による{\bf 「遠方GRB・宇宙再電離期の物理状況の理解」}という日本発のサイエンスが期待でき、その意義は非常に高い。


