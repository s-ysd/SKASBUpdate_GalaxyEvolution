%%
\subsection{突発天体研究のためのSKAへの要求} \label{transients.s3.requirements}
%%

SKAは既にデザインがおおよそ決まっており、その装置性能や運用体制に対して修正を要求することは難しい。
そのことを踏まえて本節では、現状のデザインでどの程度のサイエンスを進められるか、あるいは最低限必要な性能は何かという点について主に言及してきた。
ここではそれらについて再度簡単にまとめる。

\subsubsection{アンテナの選択}
突発天体研究の進め方は二通りあり、(1) 他波長で発見された突発天体を追観測するという方法と、(2) 新たな突発天体を求めて主体的に探査するという方法がある。
どちらの方法も重要であり、(1) では従来の望遠鏡ではなしえない感度によって、起源天体の物理の解明をめざし、(2) では、理論予測はされていてもまだ観測されていないような天体を発見し、さらには人類がまだ予想すらしていないような未知の発見を生む可能性が高い。
そして強いて言うならば、(1) は主に\skamid{}によって実施され、(2) は\skasur{}によって実施される。
また\skalow{}は装置の特色上 (1) と (2) 両方に注力できる。

\skamidsur{} はどちらも主鏡は同じものを使用するが、そのフィードや受信機系は異なり、\skamid{}は広帯域、高感度、高分解能に、\skasur{}は広い視野による探査速度に長けている。
日本の既存望遠鏡との連携という観点からは、時差のほとんどないオーストラリアに建設される\skasur{}が突発天体探査と追観測には有利といえる\footnote{ただし北半球の日本と南半球のオーストラリアでは観測範囲が重ならないことも多い。}。
現状の突発天体研究においては、起源がわかっていないものなどは300~MHzなどの低周波帯 (e.g., GCRT)、あるいは1.4~GHz帯に多く見つかっている (e.g., FRB)。
一方で超新星などの場合5~GHz帯での観測がよく行われており、\skalowmidsur{} の全てが突発天体研究にとっては重要な装置である。
しかし現状の日本の突発天体コミュニティでは、主に\skamidsur{} を用いたサイエンスの提案が多いため、現時点では主にそれらに絞って装置性能を考える。

\subsubsection{装置要求}
以上のことを踏まえて、以下で各観測パラメータについて簡単にまとめる。

\paragraph{周波数範囲}
従来発見されてきた突発天体は数十MHzという低周波からGHz帯まで幅広く発見されてきており、特別な周波数というものはない。
ただしFRBやパルサー観測では主に1.4~GHz帯で強度が強いため観測しやすく、またその周波数帯は中性水素輝線の観測で重要であるから、Lバンドを観測できる受信機の実装を最優先すべきである。
\citet{2011Natur.478...82N}によれば、中性子星連星合体で重力波とともに1.4~GHzでピークをもつ電波が放射されるので、Lバンド受信機の実装は重力波天文学においても重要である。
また超新星やGRBの観測ではより高周波帯を観測する必要があり、Cバンド受信機などの実装も急ぐべきである。

\paragraph{感度}
感度は高ければ高いだけ良い。
例えば超新星の検出数と感度との関係は\Secref{transients.s2}の原論文に書かれており、また未知の突発天体の探査に必要な感度は、例えば\Eqref{eq:transients.s3.unknowns.SNII}を満たす範囲で、コストに見合う性能を追求すればよい。

\paragraph{時間分解能}
突発天体探査においては、時間分解能の多様性が極めて重要な要素である。
これまで発見されてきた突発天体は、Crab nanoshots のようなナノ秒スケールの変動や、FRBや普通のパルサーのようなミリ秒程度の変動、またGCRTイベントのような数分スケールの変動からWJNイベントのような数時間、数日程度の変動、さらにGRB残光のような数か月、数年という変動など、あらゆる時間領域に渡っている。
したがって、それらを網羅できるような時間分解観測が必要である。
このうち、パルス観測はコンピュータによる計算コストは高いが、観測自体は容易であり、また数日以上続くような長時間変動の場合も、通常の干渉計観測を行えばよく容易といえる。
おそらく観測自体がやや難しい (面倒な) のは変動のタイムスケールが数分程度の現象であり、感度と空間分解能をいかに維持して数分スケールの変動を追うかが課題になるだろう。

\paragraph{偏波}
偏波情報は、突発現象の放射機構解明などに必須であり、その受信系は必ず実装すべきである。
また\Secref{transients.s3.frb.magnetism}に記したように、FRBを用いて銀河間磁場を解明できる可能性もある。

\subsubsection{まとめ}
SKAデザインは基本的には決められており、またその既存デザインで、日本コミュニティの期待する成果は十分に出せると見積もっている。
未知天体の探査については、その最低限の要求は\Eqref{eq:transients.s3.unknowns.SNII}で与えており、その要求を満たせば多くの超新星などが発見され、それらの研究において大きなブレイクスルーとなる。
