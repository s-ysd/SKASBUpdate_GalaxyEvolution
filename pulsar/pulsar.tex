%%%%%%%%%%%%%%%%%%%%%%%%%%%%%%%%%%%%%%%%%%%%%%%
\documentclass[a4j,twoside,11pt]{jreport}
\renewcommand{\bibname}{参考文献}
\usepackage[usenames]{color}
\usepackage{graphicx}
\usepackage{wrapfig}
\usepackage{times}
\usepackage{natbib}
\usepackage{revsymb}
\usepackage{multicol}
%\usepackage{aas_macros}
\usepackage{url}
\usepackage{amsmath,amssymb,amsfonts,bm}
\usepackage[dvipdfm,colorlinks=false,bookmarks=true,bookmarksnumbered=false,pdfborder={0 0 0},bookmarkstype=toc]{hyperref}
\setlength{\textheight}{230mm}
\setlength{\textwidth}{160mm}
\setlength{\topmargin}{0mm}
\setlength{\oddsidemargin}{0mm}
\setlength{\evensidemargin}{0mm}
\setlength{\columnseprule}{1pt}
\setlength{\columnsep}{18mm}
\setlength\bibsep{1pt}
\pagestyle{headings}


\begin{document}


\chapter{パルサー}\label{pulsar}


%\begin{figure}[t]
%\begin{center}
%\includegraphics[width=0.6\linewidth]{***.eps} 
%\caption{}
%\label{fig:cosmo_history}
%\end{center}
%\end{figure}


\section{パルサー研究の現状と未解決問題}\label{pulsar.s1}

\subsection{パルサー磁気圏}

\subsection{パルサー風}

\subsection{パルサー内部構造}

\subsection{パルサーを用いた銀河系構造探索}

\subsection{パルサーによる重力波検出}

\subsection{パルサーによる一般相対論検証}



\newpage

\section{国際SKAのサイエンス}\label{pulsar.s2}

\subsection{パルサー人口調査}

\subsection{パルサーポピュレーション}

\subsection{銀河中心のパルサー}

\subsection{球状星団のパルサー}

\subsection{パルサー磁気圏}

\subsection{パルサー風}

\subsection{内部構造と核物質の状態方程式}

\subsection{重力理論検証}

\subsection{重力波検出}

\subsection{銀河系の構造}



\newpage

\section{日本が狙うサイエンス}\label{pulsar.s3}





\begin{thebibliography}{99}

\bibitem[Chapman et al.(2012)]{2012MNRAS.423.2518C}
Chapman, E., Abdalla, F.~B., Harker, G., et al.\ 2012, MNRAS, 423, 2518 


\end{thebibliography}


\end{document}

