%%%%%%%%%%%%%%%%%%%%%%%%%%%%%%%%%%%%%%%%%%%%%%%
%%%%%%%%%%%%%%%%%%%%%%%%%%%%%%%%%%%%%%%%%%%%%%%
%%%%%%%%%%%%%%%%%%%%%%%%%%%%%%%%%%%%%%%%%%%%%%%
\setcounter{section}{1}\section{国際SKAサイエンスブックの紹介}
\label{c08.s2}

国際SKAサイエンスブックにおいて星間物質という領域区分はないが、星間物質やその周辺
(太陽圏、星周、パルサー周辺、銀河)を含む論文は既に複数出版されている。星間物質科学検討班では、今後これらの論文の解説をまとめていく予定である。表\ref{c08.s2.t1}にはその解説に代わり、星間物質のサイエンスを含む国際SKAサイエンスブックの論文をリストする。

\begin{table}
\caption{国際SKAサイエンスブックの星間物質とその周辺領域論文一覧$^*$}
\begin{center}
\small
\begin{tabular}{llp{28zw}}
\hline\hline
\noalign{\smallskip}
ID$^\dag$ & PI & Title\\
\noalign{\smallskip}
\hline
\noalign{\smallskip}
Pulsar &&\\
041 & J. Han  & Three-dimensional Tomography of the Galactic and Extragalactic Magnetoionic Medium with the SKA \\
\noalign{\smallskip}
\hline
\noalign{\smallskip}
Continuum &&\\
068 & M. Jarvis  & The star-formation history of the Universe with the SKA\\
070 & R. Beswick  & SKA studies of nearby galaxies: star-formation, accretion processes and molecular gas across all environments\\
082 & C. Mancuso  & Radio Observations of Star Forming Galaxies in the SKA era\\
085 & E. Murphy  & The Astrophysics of Star Formation Across Cosmic Time at 10 GHz with the Square Kilometer Array\\
\noalign{\smallskip}
\hline
\noalign{\smallskip}
Magnetism &&\\
094 & R. Beck  & Structure, dynamical impact and origin of magnetic fields in nearby galaxies in the SKA era\\
096 & M. Haverkorn  & Measuring magnetism in the Milky Way with the Square Kilometre Array\\
102 & C. Dickinson  & SKA studies of in situ synchrotron radiation from molecular clouds\\
106 & G. Heald  & Magnetic Field Tomography in Nearby Galaxies with the Square Kilometre Array\\
110 & T. Robishaw  & Measuring Magnetic Fields Near and Far via the Zeeman Effect\\
\noalign{\smallskip}
\hline
\noalign{\smallskip}
Cradle of Life &&\\
118 & G. Umana  & The impact of SKA on Galactic Radioastronomy: continuum observations\\
119 & J. Green  & Maser Astrometry with VLBI and the SKA\\
124 & C. Dickinson  & Studies of Anomalous Microwave Emission (AME) with the SKA\\
125 & S. Etoka  & OH masers in the Milky Way and Local Group galaxies in the SKA era\\
126 & M. Thompson  & The ionized,radical and molecular Milky Way: spectroscopic surveys with the SKA\\
166 & L. Loinard  & SKA tomography of Galactic star-forming regions and spiral arms\\
\noalign{\smallskip}
\hline
\noalign{\smallskip}
H\,\textsc{i} &&\\
128 & S. Blyth  & Exploring Neutral Hydrogen and Galaxy Evolution with the SKA\\
129 & E. de Blok  & The SKA view of the Neutral Interstellar Medium in Galaxies\\
130 & N. McClure-Griffiths  & Galactic and Magellanic Evolution with the SKA\\
131 & M. Meyer  & Connecting the Baryons: Multiwavelength Data for SKA H\,\textsc{i} Surveys\\
134 & R. Morganti  & Cool Outflows and H\,\textsc{i} absorbers with SKA\\
139 & R. Oonk  & The Physics of the Cold Neutral Medium: Low-frequency Radio Recombination Lines with the Square Kilometre Array\\
\noalign{\smallskip}
\hline
\noalign{\smallskip}
Synergies &&\\
152 & G. Fuller  & Star and Stellar Cluster Formation: ALMA-SKA Synergies\\
156 & R. Paladino  & Synergies between SKA and ALMA: observations of Nearby Galaxies\\
157 & J. Antoniadis  & Multi-wavelength, Multi-Messenger Pulsar Science in the SKA Era\\
161 & J. Wagg  & Enabling the next generation of cm-wavelength studies of high-redshift molecular gas with the SKA\\
169 & V. Nakariakov  & Solar and Heliospheric Physics with the Square Kilometre Array\\
\noalign{\smallskip}
\hline
\noalign{\smallskip}
\multicolumn{2}{l}{\small $^{*}$ArXiv未投稿含む} & \\
\multicolumn{2}{l}{\small $^\dag$ PoS(AASKA14) ID} & \\
\end{tabular}\label{c08.s2.t1}
\end{center}
\end{table}
