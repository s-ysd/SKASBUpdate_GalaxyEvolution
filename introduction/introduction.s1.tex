%%%%%%%%%%%%%%%%%%%%%%%%%%%%%%%%%%%%%%%%%%%%%%%
%%%%%%%%%%%%%%%%%%%%%%%%%%%%%%%%%%%%%%%%%%%%%%%
%%%%%%%%%%%%%%%%%%%%%%%%%%%%%%%%%%%%%%%%%%%%%%%
\setcounter{section}{0}\section{SKA計画の概要}
\label{c01.s1}

%%%%%%%%%%%%%%%%%%%%%%%%%%%%%%%%%%%%%%%%%%%%%%%
%%%%%%%%%%%%%%%%%%%%%%%%%%%%%%%%%%%%%%%%%%%%%%%
\subsection{はじめに}
\label{c01.s1.ss1}

\paragraph{概要}

SKA (Square Kilometre Array) 計画とは、2020年代にアフリカとオセアニアに電波望遠鏡を建設する計画である。日米欧らが推進するミリ波・サブミリ波望遠鏡ALMAでは観測できない、センチ波・メートル波を網羅し、この波長の観測が極めて重要な科学的課題に取り組む。高感度、高分解能、広視野、広帯域という科学要求を実現するため最先端の技術を結集し、それでいて安価で耐久性のある望遠鏡を建設する。計画は第1期(SKA1)と第2期(SKA2)に分かれ、2015年現在は第1期の建設前段階にある。SKAはこの波長で世界唯一の国際大型計画である。11ヶ国が出資し、SKA機構が運営している。そして20ヶ国100機関以上の組織が設計に関わっている。日本では160名以上(2015年1月現在)の科学者・技術者が参加する日本SKAコンソーシアムを母体に、科学検討班が活動の中核を成し、SKA計画への参入に向けて準備を進めてきている。

%%%%%%%%%%%%%%%%%%%%%%%%%%%%%%%%%%%%%%%%%%%%%%%
%%%%%%%%%%%%%%%%%%%%%%%%%%%%%%%%%%%%%%%%%%%%%%%
\subsection{科学的な動機}
\label{c01.s1.ss2}

\paragraph{センチ波・メートル波とは}

SKAが網羅する電波は、周波数分類で言えば超短波(VHF、30--300 MHz)、極超短波(UHF、300--3000 MHz)、そしてセンチ波(SHF、3--30 GHz)である。古くから身近な電磁波として生活に密着しており、現代ではラジオやテレビの放送、携帯電話や無線LANなどの通信、非接触ICカード、航空レーダーや誘導装置など、人類の文明になくてならない存在である。天文学においても、センチ波・メートル波は地球の大気吸収をほとんど受けないため、地上から宇宙を眺めることのできる貴重な電磁波である。1931年のKarl Janskyによる宇宙電波の発見を最初に、1951年に相次いだ中性水素原子21cm輝線の発見、1967年のSusan Jocelyn Bellによるパルサーの発見、1964年のArno Allan PenziasとRobert Woodrow Wilsonによる宇宙背景放射の発見など、センチ波・メートル波観測によって数多くの天文学上重要な発見があった。

\paragraph{センチ波・メートル波の可能性}

この3桁にも及ぶ波長帯の電波を観測する意義は、それでこそ果たせる天文学的使命があるからに他ならない。それは大まかにはつぎの3つに集約される。
\begin{itemize}
\item[★] {\bf 特定物質を観測できる:} この波長帯では中性水素原子の超微細構造線を観測できる。この輝線や吸収線は宇宙の物質とその分布を探る確立した方法であり、宇宙再電離の早期を直接観測する方法としては事実上唯一の方法である。この波長帯では星間分子雲のトレーサーであるOH分子輝線や、生命に関係するグリシン・アラニンなどの輝線も調べることができる。すなわち、太古に中性だった宇宙が再イオン化する過程を探り、銀河史や化学進化史を描き出し、物質の相転移や星・惑星・生命が作られていく現場に迫っていくことができる。
\item[★] {\bf シンクロトロン放射を観測できる:} この波長帯はシンクロトロン放射の観測に有利である。荷電粒子が磁場中で加速されて放たれるこの放射は、ほとんどの活動的・突発的な天体ならびに天体現象に伴うため、パルサー、ブラックホール、銀河、銀河団などの形成史や分布、宇宙論に迫ることができる。天体の距離や軌道を極高精度に測ることができれば、重力波によって生じる時空振動を調べることができ、一般相対論の検証も可能である。現実的にこれができるのは、パルサーの距離と周期を極高精度に測ることが可能なこの帯域に限られる。放射スペクトルからは、宇宙線の基礎物理過程も研究することができる。
\item[★] {\bf 偏波を観測できる:} この波長帯は偏波の観測にも有利である。シンクロトロン放射と、磁場中を通り抜ける際に生じる直線偏波のファラデー回転を、同時に高精度に観測することができる。さらに、ゼーマン効果によって分離した輝線も円偏波あるいは直線偏波で観測することができる。これらは宇宙空間の磁場を3次元的に探ることのできる確立した方法であり、事実上最も有効な方法である。星間、銀河、銀河団の磁場の誕生と成長、降着円盤やジェットのメカニズム、銀河間磁場の宇宙論的な重要性に迫ることができる。磁場と密接に関係する乱流物理の検証も可能である。
\end{itemize}

\paragraph{SKAの科学目標}

上記のセンチ波・メートル波のユニークな優位性に加え、21世紀の時代にこの帯域でしか果たせないこと、この帯域が極めて重要な役割を担うこと、さらには他波長研究との相補性を考慮して、SKA計画ではいくつかの主要な科学目標が掲げられている。それは
\begin{itemize}
\item[★] 銀河進化と宇宙論(Galaxy evolution, cosmology and dark energy)
\item[★] 重力理論の検証(Strong-field tests of gravity using pulsars and black holes)
\item[★] 宇宙磁場の起源と進化(The origin and evolution of cosmic magnetism)
\item[★] 宇宙の暗黒時代(Probing the Cosmic Dawn)
\item[★] 生命のゆりかご(The Cradle of life)
\item[★] 未知の発見(Flexible design to enable exploration of the unknown)
\end{itemize}
である。2004年にはこれらの領域を中心に、SKA計画の最初のサイエンスブック「Science with the Square Kilometre Array」が編纂され出版された\footnote{
Science with the Square Kilometer Array, 2004, New Astronomy Review (Elsevier, Amsterdam), eds. C. Carilli \& S. Rawlings}。

%%%%%%%%%%%%%%%%%%%%%%%%%%%%%%%%%%%%%%%%%%%%%%%
%%%%%%%%%%%%%%%%%%%%%%%%%%%%%%%%%%%%%%%%%%%%%%%
\subsection{科学要求と技術要求}
\label{c01.s1.ss3}

\paragraph{科学要求} 

SKA計画の原点は、中性水素原子(H\,\textsc{i}) 21cm線を$1''$分解能でマッピングするというHydrogen Array計画である。これは1991年IAUシンポジウムにおいて英国ジョドレルバンク天文台のPeter Williamsによって提案された。その後、センチ波メートル波の様々な可能性が合流しながら、集光面積に焦点をおきSquare Kilometre Array (SKA)と呼ばれるようになった。SKAとなってからは、上記に挙げた多種の科学目標を達成するためにどのような性能が必要かについて多くの議論がなされた。そして、大枠として次のような性能が必要であると結論付けられた。
\begin{itemize}
\item[★] 既存装置として最大級の米国の旧VLA対比で1桁以上改善された高い感度(50倍以上)
\item[★] この波長帯でかつてないほど幅広い帯域(50 MHz -- 30 GHz)
\item[★] 全天の走査を容易にする広い視野(GHz帯で200平方度・満月2000個分)
\item[★] VLA対比で1桁以上改善された高い空間分解能(短波長側で0.1秒角以下)
\end{itemize}
後述する国際SKAサイエンスブック2015では、それぞれのサイエンステーマについてより具体的な科学要求の試みがなされている。本書でもその内容を紹介していく。

\paragraph{技術要求}

以上の基本的な科学要求を達成するためには、アンテナの総有効面積が1平方キロメートル程度、最大基線長が数1000 kmという巨大な観測局群が必要である。これはオーストラリア大陸全部の大きさに相当する、まさに地球規模・人類史上最大の望遠鏡である。またこのような観測局群の他に、SKAの中心にはエクサフロップス性能(2012年の世界No.1スーパーコンピューターである京の100倍の性能)のスーパーコンピュータを設置したデータセンター、またデータの送受信や電力を供給する広大なネットワークが必要である。広視野と広帯域の実現については、従来の方式とは異なる全く新しい技術の確立が不可欠である。

\begin{figure}[tbp]
\begin{center}
\includegraphics[width=\linewidth]{introduction/c01.s1.f1.eps}
\caption{天の川を探るSKAの構想図。アフリカとオセアニアのアンテナ群が合成されてある。
}\label{c01.s1.f1}
\end{center}
\end{figure}

\paragraph{要求の実現にむけて}

このような規模の要求を一国で達成することは困難であるため、世界各国が協力をすることでその達成を目指すことにした。アンテナは低周波用と高周波用の2種類(広視野高周波を加えると3種類)を建設することに決めた。建設地は広大な平地が確保できかつ電波障害の最も少ない南アフリカとその周辺国、そしてオーストラリアとニュージーランドに分散させて建設することに決めた。そして工期は第1期(SKA1)と第2期(SKA2)とに分け、SKA1は最終構成の10\%を2016年から建設し2020年から運用、SKA2は残り90\%を2020年から建設し、2025年頃から最終構成で運用することに決めた。これがSKA計画の概要である。図\ref{c01.s1.f1}には、天の川を探るSKAの想像図をSKAのホームページより引用する。
