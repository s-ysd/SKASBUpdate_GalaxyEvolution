%%%%%%%%%%%%%%%%%%%%%%%%%%%%%%%%%%%%%%%%%%%%%%%
%%%%%%%%%%%%%%%%%%%%%%%%%%%%%%%%%%%%%%%%%%%%%%%
%%%%%%%%%%%%%%%%%%%%%%%%%%%%%%%%%%%%%%%%%%%%%%%
\setcounter{section}{2}\section{我が国のSKA計画の歩み}
\label{c01.s3}

%%%%%%%%%%%%%%%%%%%%%%%%%%%%%%%%%%%%%%%%%%%%%%%
%%%%%%%%%%%%%%%%%%%%%%%%%%%%%%%%%%%%%%%%%%%%%%%
\subsection{これまでの活動状況}
\label{c01.s3.ss1}

\paragraph{我が国の参加の意義}

我が国がこのような世界規模の計画に金銭的・技術的に参画することは、イニシアチブを確保して研究を有利に進めるだけでなく、計画を通じた科学・技術の発展や人材の育成に大きな意義をもつ。世界中の研究者が英知を結集してSKA計画を進めている中で、科学・技術の先進国として、日本が世界から大きな期待を持たれていることも事実である。日本学術会議はその提言「第22期学術の大型研究計画に関するマスタープラン2014」の中で重点大型研究計画としてSKAを選んでいる。

\paragraph{日本SKAコンソーシアム}

日本では2008年に日本SKAコンソーシアム(SKA-JP)が結成され、 研究者レベルでの活動が行われている。現在160名を超える科学者・技術者が参加している。運営はコンソーシアム代表(杉山直、名古屋大学)と副代表(中西裕之、鹿児島大学、ならびに市來浄與、名古屋大学)、資金獲得担当(今井裕、鹿児島大学)ならびに広報担当(市來浄與、名古屋大学)が担っている。活動の中核は次に紹介するSKA-JP科学検討班が担っている。また技術開発を目指すSKA-JP技術検討班や産業界と連携をするための産業フォーラムも組織されている。

\paragraph{科学検討班}

SKA-JPには科学検討班が設置され、SKA計画に関する様々な活動の中核を成している。運営は代表(高橋慶太郎、熊本大学)と副代表(竹内努、名古屋大学)が担っている。科学検討班は次の6組織あり、高赤方偏移については本書執筆のため3グループを形成した。
\begin{itemize}
\item[★] 高赤方偏移(代表:平下博之、ASIAA)
\begin{itemize}
\item[●] 宇宙論(代表:山内大介、東京大学)
\item[●] 再電離(代表:市來淨與、名古屋大学)
\item[●] 銀河進化(代表:竹内努、名古屋大学)
\end{itemize}
\item[★] パルサー(代表:高橋慶太郎、熊本大学)
\item[★] 宇宙磁場(代表:赤堀卓也、鹿児島大学)
\item[★] 位置天文(代表:今井裕、鹿児島大学)
\item[★] 星間物質(代表:半田利弘、鹿児島大学)
\item[★] 突発天体(代表:青木貴弘、早稲田大学)
\end{itemize}
それぞれの組織は数10名程度のメンバーによって構成され、自立的に活動をしている。本書の執筆は各科学検討班メンバーらが行った。著者一覧を各章の末尾に掲載する。なお、SKA-JPの設立後早期から活動を蓄積している班もあれば、本書の執筆のために数ヶ月前に結成された班もある。そのため章ごとに検討進度にかなりの違いがあることを予めご理解頂きたい。

%%%%%%%%%%%%%%%%%%%%%%%%%%%%%%%%%%%%%%%%%%%%%%%
%%%%%%%%%%%%%%%%%%%%%%%%%%%%%%%%%%%%%%%%%%%%%%%
\subsection{本書について}
\label{c01.s3.ss2}

\paragraph{編纂の経緯}

本書の製作には3つの目的がある。1つ目に、我が国のSKA計画への参入について学会的議論を醸成するために、非専門家、たとえば電波天文学を専門としない科学者や若手研究者にもSKAの科学的課題の本質を理解して頂く目的がある。その達成のために、本書ではSKAによって解明を目指す課題の基礎を比較的丁寧にまとめた。2つ目に、SKA計画においては日本は後発にあるため、世界の情勢を把握する目的がある。特にSKA計画の可能性と実現性を見定める目的がある。その達成のために、本書では国際SKAサイエンスブック2015の内容を紹介した。3つ目に、日本がSKA計画に参加した場合の意義を明確にする目的がある。その達成のために、本書では我が国がSKAで果たせるであろう独自性をまとめた。各章はこれらの3項目をそれぞれ教科書的、総説的、そして白書的にまとめている。

\paragraph{本書の範囲}

SKA計画には多岐にわたる科学目標があるため、その全てを網羅し解説をすることには困難を伴った。日本SKAコンソーシアムには様々な分野の専門家が集まっており、各自が各々の分野の報告をまとめることによって、できるだけ幅広い解説ができるように努めた。しかし、例えば宇宙生命分野など、本書で網羅出来なかった科学的課題もいくつかある。それらについては、出版の予定される国際SKAサイエンスブックをぜひ参照されたい。その他本書の執筆の参考としている主要なSKA文書を表\ref{c01.s3.t1}にまとめた。

\begin{table}
\caption{SKA重要文書一覧とそのリンク}
\begin{center}
\footnotesize
\begin{tabular}{p{48zw}}
\hline\hline
\noalign{\smallskip}
★ Phase 1 Baseline Design: 
\url{http://astronomers.skatelescope.org/ska1/}\\
\noalign{\smallskip}
★ Phase 2 Predicted Design: 
\url{http://astronomers.skatelescope.org/ska2/}\\
\noalign{\smallskip}
★ Phase 1 Concept of Operations (revision B): 
\url{http://astronomers.skatelescope.org/wp-content/uploads/2014/03/SKAConOpsRevB.pdf}\\
\noalign{\smallskip}
★ Phase 1 Science Priority Outcome: 
\url{http://astronomers.skatelescope.org/wp-content/uploads/2014/10/SKA-TEL-SKO-0000122-SCI-REQ-RE-01-SKA1SciencePrioritiesOutcome.pdf}\\
\noalign{\smallskip}
★ Phase 1 Imaging Science Performance: 
\url{http://astronomers.skatelescope.org/wp-content/uploads/2014/06/SKA1_Science_Performance_RevA_draft5.pdf}\\
\noalign{\smallskip}
★ Phase 1 Level 1 (System) Requirements Specification (revision 5): 
\url{http://astronomers.skatelescope.org/wp-content/uploads/2015/02/SKA-TEL-SKO-0000008-AG-REQ-SRS-Rev05-SKA1_Level_1_System_Requirement_Specification-signed.pdf}\\
\noalign{\smallskip}
★ Phase 1 Level 0 (Science) Requirements (latest draft): 
\url{http://astronomers.skatelescope.org/wp-content/uploads/2014/06/SKA1-Level0-Requirements.pdf}\\
\noalign{\smallskip}
★ Phase 1 Scientific Use Cases (latest draft): 
\url{http://astronomers.skatelescope.org/wp-content/uploads/2014/05/SKA-SCI-USE-001-G_Science_use_cases-signed1.pdf}\\
\noalign{\smallskip}
★ SWG Assessment Workshop Summary: Continuum: 
\url{https://indico.skatelescope.org/getFile.py/access?resId=0&materialId=4&confId=261}\\
\noalign{\smallskip}
★ SWG Assessment Workshop Summary: Cosmology: 
\url{https://indico.skatelescope.org/getFile.py/access?resId=0&materialId=3&confId=273}\\
\noalign{\smallskip}
★ SWG Assessment Workshop Summary: Cradle of Life: 
\url{https://indico.skatelescope.org/getFile.py/access?resId=0&materialId=3&confId=266}\\
\noalign{\smallskip}
★ SWG Assessment Workshop Summary: EoR and the Cosmic Dawn: 
\url{https://indico.skatelescope.org/getFile.py/access?resId=0&materialId=2&confId=259}\\
\noalign{\smallskip}
★ SWG Assessment Workshop Summary: HI Galaxy: 
\url{https://indico.skatelescope.org/getFile.py/access?resId=0&materialId=3&confId=262}\\
\noalign{\smallskip}
★ SWG Assessment Workshop Summary: Magnetism: 
\url{https://indico.skatelescope.org/getFile.py/access?resId=0&materialId=1&confId=274}\\
\noalign{\smallskip}
★ SWG Assessment Workshop Summary: Pulsars: 
\url{https://indico.skatelescope.org/getFile.py/access?resId=0&materialId=3&confId=260}\\
\noalign{\smallskip}
★ SWG Assessment Workshop Summary: Transients: 
\url{https://indico.skatelescope.org/getFile.py/access?resId=0&materialId=2&confId=275}\\
\noalign{\smallskip}
★ Phase 1 System Baseline Design (Corrections): 
\url{https://www.skatelescope.org/wp-content/uploads/2014/11/SKA-TEL-SKO-0000002-AG-BD-DD-01-SKA1_System_Baseline_Design_Miscellaneous_Corrections.pdf}\\
\noalign{\smallskip}
★ Phase 1 System Baseline Design (2013 March 12): 
\url{http://www.skatelescope.org/wp-content/uploads/2012/07/SKA-TEL-SKO-DD-001-1_BaselineDesign1.pdf}\\
\noalign{\smallskip}
★ Project Execution Plan: 
\url{http://www.skatelescope.org/uploaded/38221_SKA_Project_Execution_Plan.pdf}\\
\noalign{\smallskip}
★ Phase 1 Design Reference Mission (DRM version 2.0): 
\url{http://www.skatelescope.org/uploaded/18714_SKA1DesRefMission.pdf}\\
\noalign{\smallskip}
★ Phase 2 Design Reference Mission (DRM version 1.0): 
\url{http://www.skatelescope.org/uploaded/3517_DRM_v1.0.pdf}\\
\noalign{\smallskip}
\hline
\end{tabular}\label{c01.s3.t1}
\end{center}
\end{table}

\paragraph{各章の内容}

本書では宇宙再電離 (\S \ref{EoR})、宇宙論 (\S \ref{cosmology})、銀河進化 (\S \ref{galaxy})、パルサー (\S \ref{pulsar})、宇宙磁場 (\S \ref{magnetism})、近傍宇宙時空計測 (\S \ref{astrometry})、星間物質 (\S \ref{ISM})、突発天体 (\S \ref{transients}) のサイエンスについて触れる。

